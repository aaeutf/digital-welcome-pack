\subsection{历史及政治}

法国历史和政治制度深刻影响着社会生活、文化氛围、法律环境和日常规则。对于留学生而言,了解法国的历史脉络和政体结构,不仅有助于更好地适应当地生活、理解社会现象和政策变化,也有助于跨文化交流、学业研究和个人成长。只有深入理解法国的历史与政治,才能真正融入法国社会,避免文化误解,更好地把握学习和生活的机遇。

\subsubsection{历史}

\paragraph{史前} 法国地区在旧石器时代就有人类活动,距今约4万年前的克罗马农人遗址即位于法国。著名的拉斯科洞穴壁画(Lascaux)展现了史前人类的艺术创造力。新石器时代,农业和畜牧业逐步发展,形成了早期的部落社会。

\paragraph{早期} 公元前6世纪至公元前1世纪,法国地区主要由高卢人(凯尔特人)居住,分为多个部落。高卢人在文化、宗教和社会组织上有独特传统。公元前5世纪,希腊人在地中海沿岸建立马赛等殖民地。公元前1世纪,罗马人入侵高卢,最终在凯撒的高卢战争后将其纳入罗马帝国版图。

\paragraph{罗马时期} 公元前1世纪至公元5世纪,高卢成为罗马帝国的一个重要行省。罗马带来了道路、城市、法律和拉丁语言,促进了经济和文化发展。基督教在此时期逐渐传播。随着西罗马帝国衰落,蛮族入侵,罗马统治结束。

\paragraph{中世纪} 5世纪末,法兰克人建立法兰克王国,查理曼大帝于8世纪统一西欧,成为“欧洲之父”。843年《凡尔登条约》后,西法兰克成为法国雏形。中世纪法国经历了封建割据、十字军东征、百年战争(1337-1453)等重大事件。随着中央集权的逐步加强,法国逐步成为欧洲强国。

\paragraph{文艺复兴时期} 15世纪末至17世纪初,法国进入文艺复兴时期,受意大利影响,艺术、文学、科学和思想空前繁荣。弗朗索瓦一世大力支持文艺复兴运动,卢浮宫等建筑在此时期扩建。宗教改革和宗教战争(如胡格诺战争)也深刻影响了法国社会。文艺复兴为法国现代国家的形成奠定了文化和制度基础。

\paragraph{近代前期} 18世纪末,法国社会矛盾激化,启蒙思想传播,为大革命埋下伏笔。此时法国仍为波旁王朝统治,社会分为贵族、教士和平民三级。

\paragraph{法国大革命时期} 1789年法国大革命爆发,攻占巴士底狱成为象征性事件。大革命推翻了君主专制,建立了法兰西第一共和国。此时期经历了吉伦特派、雅各宾派的斗争,罗伯斯庇尔主导的“恐怖统治”,以及热月政变等重大事件。大革命深刻改变了法国社会结构,确立了自由、平等、博爱的理念。

\paragraph{近现代} 大革命后,拿破仑于1804年建立法兰西第一帝国。其后法国多次经历王朝更替和政权更迭:1815年滑铁卢战败后波旁王朝复辟,1830年七月革命建立法兰西第二共和国,1851年拿破仑三世建立法兰西第二帝国,1870年普法战争后成立第三共和国。二战期间法国被德国占领,南部成立维希政权。战后戴高乐领导成立第四共和国,1958年建立第五共和国,延续至今。

\subsubsection{现状}

法国第五共和国实行半总统制共和政体,兼具总统制和议会制的特点,其核心设计和原则来源于法国18世纪哲学家孟德斯鸠提出的三权分立——即立法、行政和司法三种国家权力互相制衡。

\textbf{行政权} 由总统和总理共同掌握。总统(Président de la République)为国家元首,由全民直选产生,任期五年,拥有较大权力,包括任命总理、主持部长会议、解散国民议会、发布法令、主导外交和国防等。总统在外交和国防事务中拥有主导权。总理(Premier Ministre)为政府首脑,由总统任命,需获得国民议会信任,负责政府日常行政管理和政策执行。总统与总理的权力分工体现为“二元执行权”,在总统和议会多数党不同时,可能出现“共治”(cohabitation)现象。

\textbf{立法权} 由两院制的议会行使,包括国民议会(Assemblée Nationale)和参议院(Sénat)。国民议会议员由普选产生,参议员由间接选举产生。国民议会拥有更大立法权力,可推翻政府。

\textbf{司法权} 独立于行政和立法系统,主要由普通法院(Tribunal judiciaire)、行政法院(Tribunal administratif)和宪法委员会(Conseil constitutionnel)等组成,保障法律的公正实施和宪法的权威。

\textbf{地方自治} 法国实行地方分权,设有大区、省、市等三级地方政府,地方议会由选举产生,享有一定自治权。

\textbf{价值观与宪法保障} 法兰西共和国的价值观暨座右铭是“自由,平等,博爱(Liberté, Égalité et Fraternité)”,这三个词会出现在法国的很多建筑和标记中。法国宪法保证公民的人权和自由,包括言论自由、信仰自由、著作和出版自由、集会及结社自由等。法国宪法经历了两个多世纪的发展和改良,已成为西方民主政治的典范。其核心价值和原则,对自由、民主和人权的追求,也成为了法兰西文化的重要组成部分。

\textbf{当前政局} 现任总统马克龙(Emmanuel Macron)是“一起为了共和国”(Ensemble pour la République)党(中间偏右派)的代表。2024年6月欧洲议会选举中,“一起为了共和国”落败于极右翼“国民联盟”(Rassemblement national),随后马克龙宣布解散国民议会,并于2024年7月7日完成新一轮议员选举。目前左派新人民阵线(Nouveau Front Populaire)获得国民议会最多议席,但无任何政党或联盟取得绝对多数。法国政坛呈现多党分裂格局,主要党派包括中间派的“一起为了共和国”、极右翼的“国民联盟”、左翼联盟“新人民阵线”以及传统右翼“共和党”等。极右翼势力近年来快速崛起,反映出社会对移民、治安、经济等问题的担忧加剧。联合政府成为常态,政策推动和议会合作难度加大。社会层面,法国近年来频繁爆发大规模罢工、抗议和社会运动,反映出社会对养老金改革、物价上涨、社会不平等等问题的强烈不满。社会分裂和极化趋势明显,政治协商和社会治理面临较大挑战。

下一次总统选举将在2027年举行。
