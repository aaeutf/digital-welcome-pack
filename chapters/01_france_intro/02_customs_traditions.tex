\subsection{风俗及传统}

了解法国的风俗与传统,有助于留学生更好地适应当地生活、融入社会、避免文化误解,并提升跨文化交流能力。

\subsubsection{日常社交与礼仪}
法国人热爱社交,常在酒吧、咖啡馆、餐馆聚会,或邀请朋友到家中做客。日常见面多用“吻面礼”(faire la bise)或握手,熟人间常用亲切称呼。参加聚会、生日、婚礼等活动时,通常会带花束、红酒、巧克力等小礼物作为伴手礼。法国人注重礼貌,常用“您好”(Bonjour)、“谢谢”(Merci)、“请”(S'il vous plaît)等礼貌用语。

\subsubsection{餐桌礼仪}
用餐时讲究餐桌礼仪,如不随意离席、不大声喧哗、不用手直接拿食物(面包除外),餐具摆放有讲究。用餐结束时刀叉应并排放于盘中。

\subsubsection{宗教与世俗}
法国主要宗教有基督教(天主教、新教、东正教)、伊斯兰教和犹太教。自1905年起,法国实行政教分离政策,国家对所有宗教保持中立,无官方宗教。宗教多元化是法国社会的重要特征,但近年来也面临一定挑战。

\subsubsection{时间观念与作息}
法国保留夏令时和冬令时,夏令时期间与中国时差6小时,冬令时差7小时。每年3月最后一个周日进入夏令时,10月最后一个周日恢复冬令时。法国人重视生活节奏,午餐和晚餐时间较固定,商店周日多休息。

\subsubsection{传统节日与假日}
法国有丰富的法定节假日,包括:
\begin{itemize}
  \item \textbf{元旦}(Nouvel An,1月1日)
  \item \textbf{复活节}(Pâques,3-4月,日期每年不同)
  \item \textbf{劳动节}(Fête du Travail,5月1日)
  \item \textbf{二战停战日}(Victoire 1945,5月8日)
  \item \textbf{耶稣升天节}(Ascension,5月中旬,日期每年不同)
  \item \textbf{圣灵降临节}(Pentecôte,5-6月,日期每年不同)
  \item \textbf{国庆节}(Fête Nationale,7月14日)
  \item \textbf{圣母升天节}(Assomption,8月15日)
  \item \textbf{诸圣节}(Toussaint,11月1日)
  \item \textbf{一战停战日}(Armistice 1918,11月11日)
  \item \textbf{圣诞节}(Noël,12月25日)
\end{itemize}
节日当天通常会有家庭聚会、游行、宗教仪式等活动。部分节日如复活节、圣诞节有特殊美食和民俗。

\subsubsection{常见禁忌与注意事项}
避免在公共场合大声喧哗、随地吐痰、插队等。与人交谈时避免涉及宗教、收入、年龄等私人话题。法国人重视隐私和个人空间。

\subsubsection{文化多样性与包容性}
法国社会多元包容,尊重不同文化背景。留学生应保持开放心态,尊重当地风俗,同时也可自信展示自身文化。
