% 抵法初期行政手续
\section{抵法初期行政手续}

在法国,办理行政手续是每一位留学生在法生活印象深刻的必修课,需要大家给予一定的\textbf{耐心、信心和同情心}。

刚刚抵法后的行政手续办理逻辑如下:
\begin{itemize}
    \item 办理一张预付费手机卡,拥有法国手机号
    \item 确定可收取信件的固定住址,签订住房合同
    \item 凭借住房证明方可办理法国银行卡、办理签证生效OFII
    \item 凭借办理银行卡时获得的账户信息(RIB/IBAN)方可办理住房补助(CAF)、通讯手机卡、网络盒子、水电账户、学生保险等(注意无需等银行卡到手)
    \item 凭借法国手机号方可顺利办理所有其他手续(收验证码等)
    \item 凭借签证生效OFII、银行卡等注册Ameli(医保)账号、申请Vitale卡(医保卡)
\end{itemize}

行政参考:\href{https://www.ecentime.com/article/demarches-etudiant-arrivant}{ECENTIME: 法国行政手续,接好这份抵法后各种手续的办理顺序整理!}
 
\subsection{签证换居留证OFII}
在法居留证明是办理一切行政手续的必须材料。对于第一年的学生来说是护照页上贴的OFII,第一年过后是居留卡Titre de Séjour。

相信大家来法国前一定已经持有申根-法国签证。如果你持有的是6-12个月的长期签证,需在抵达法国后三个月内到法国移民局办理“签证生效手续”,又称OFII。 这一行政步骤是强制性的,以保证你在法国的法律地位。未能完成此步骤将导致非正常移民身份,您可能将无法越过申根边境,并会影响你办理后续的其他行政手续或享受法国福利(如CAF申请长居等)。

自2019年起,本业务可在线办理。办理业务需要准备法国住房合同和印花税邮票,后者可在直接在线购买或在附近烟草店Tabac购买,或线上购买。

具体办理流程可自行前往\href{https://administration-etrangers-en-france.interieur.gouv.fr}{法国移民局网站}办理:,网页右上角有选项可切换语言。

OFII参考:\href{https://www.ecentime.com/article/tuto-ofii}{ECENTIME: 法国签证生效手续怎么申请?OFII可以在网上办理啦!}
 
\subsection{银行账户}

法国以信用卡为主,不通用借记卡。主流的信用卡厂家是 Carte Blue/VISA/MasterCard,主要银行有BNP Paribas, Société Générale, Caisse Epargne, LCL, Crédit Agricole 等。银行开户要求提供住房合同,所以务必先确认自己的住房合同无误后再去办理。除此以外,还需携带护照和学生注册证明。有些银行会在学校刚开学时提供便捷服务和福利,对于学生有免年费的优惠政策,具体可以在开户时详细咨询。

开户的过程中,一定要咨询所办理银行卡的消费额度(Plafond),以及每周的取现限额。此外,如果有办理住房保险的需求,也可以直接向银行的账户顾问(Conseiller)咨询相关服务。建议同时办理银行卡与支票本。办妥当天可获得账户信息(RIB/IBAN,重要!)用于办理后续手续。开户后将用过信件在登记的固定住址分别收取银行卡和密码,办理时长为自办理日起预计 2-3星期用于制卡、邮寄。

法国是世界上为数不多的保留支票(chèque)作为主要支付手段的国家。在法国,因为现金的管制很严格,所以大额的交易一般都是用支票支付的,比如学费、房租、押金,甚至不少公司实习期间的工资也会开支票。 关于开支票付费与存支票收款,我们在网上为大家找了几篇讲解详细的攻略供大家参考。

银行参考:
\begin{itemize}
    \item \href{https://www.ecentime.com/article/creation-compte-bancaire}{ECENTIME: 法国银行开户,最全攻略敬请查收}
\end{itemize}

\subsection{手机}

法国有四家主要通讯运营商:Bouygues, SFR, Orange, Free. 法国的电话卡资费套餐通常都包含接打、短信和欧盟通用的上网流量。关于各家的优惠套餐,建议大家实时关注。

虽然去银行开户后便可凭借账户信息RIB/IBAN和其他材料去运营公司办理固定手机卡,但是由于开户和收取电话卡前后至少需等待2-5工作日不等,强烈推荐同学们落地后手中至少有一张预付费临时电话卡(Carte Prépayée)或者其他可以在法使用一周左右的手机卡,否则落地后无法立刻拥有固定手机卡与家人联系,银行开户时也需要预留手机号,且银行开户后去办手机卡需要等待邮寄。

出国旅游时,每到一个国家,都会收到运营商发出的关于当前国家的资费短信(收短信都是免费哒),告知打电话,发短信,用流量等所需费用。欧盟境内一般有一部分共享流量,建议大家提前了解自己的话费套餐是否包含出国项目,尤其是流量。如果不包含,某些运营商提供临时漫游流量套餐等,最好提前购买。

通讯参考:
\begin{itemize}
    \item \href{https://www.ecentime.com/article/forfait-mobile-france}{ECENTIME: 法国电信服务大盘点}
    \item \href{https://www.ecentime.com/article/carte-sim}{ECENTIME: 办手机卡的那些事儿}
\end{itemize}

\subsection{网络}

在法国,家庭网络的使用一般是通过一种三网合一的网络盒子(Box)完成的,可以同时提供电信网络(固定电话)、计算机网络和有线电视网络。网络技术一般分为传统的同轴电缆(ADSL),或者更高速的光纤(Fibre)。目前,法国常见的网络运营商有Orange/SFR/Bouygues/Free,分别拥有自己的网络套餐。如果住宿不包含网络的话,需要同学们自行选择一家运营商开始签订网络套餐。

办理时,首先需要你已经办理好银行开户,持有账户信息(RIB或者IBAN)。其次,你需要知道所在房间是否曾经接入过某个运营商来决定是需要架设线路还是激活休眠线路。这个信息就需要大家从房东那里询问,或者在以上4家主流运营商网站上通过各自数据库来查找。

如果找不到所在住所的网络线路信息,那么有大几率这个住所还没有架设网络线路,所以需要选择运营商和套餐,预约修理工上门架设,这个过程会收费且需要预约。

如果查找到或者已知从前房子已经办理过网络,那么仅需网上办理激活,同时需要提供上一任使用该线路的住客姓氏。一般来说,可通过房东获得或者以上几家运营商为争夺客户会主动帮助查找上任住客信息。

网络参考:
\begin{itemize}
    \item \href{https://www.ecentime.com/article/network-box-in-france}{ECENTIME: 网盒,法国BOX办理没头绪?从办理到使用全解答!}
\end{itemize}

\subsection{保险}
\subsubsection{社会医疗保险}

法国的医保体制由两部分组成:基本医疗保险以及补充险。基本医疗保险由法国医保部门Assurance maladies负责任何在法国拥有合法拘留权的人都包括在内。费用视你的收入而定。基本险一般报销各类诊疗费用的60\%-70\%。余下部分的报销就由补充险赔付。

法国学生医疗保险制度将进行改革,新推出的CVEC(La Contribution Vie Etudiante et de Camps)取代了原来的学生社会保险(La Sécurité Sociale Etudiante), 以90欧/年获得医保卡(Carte Vitale), 享受医疗保障包括门诊、就医、买药等最高70\%的报销额度。购买可通过\href{https://www.messervices.etudiant.gouv.fr}{网站}进行。付款成功后下载保留注册证明,在去学校注册时需要提供。购买成功后需要要在社保网站注册生效。

法国行政效率较为缓慢,可以通过电话沟通加快获得Carte Vitale和激活ameli账户的速度。\href{https://www.ameli.fr/paris/assure/english-pages}{电话参考}:09 74 75 36 46。

所需材料:学校注册证明、护照、学生签证、出生公证、居留证/OFII、银行账户信息RIB。

社保参考:\href{https://www.ecentime.com/article/assurance-maladie}{ECENTIME: 法国医保、学生医保不会办理?保姆级申请攻略你值得拥有!}
 
\subsubsection{辅助保险Mutuelle – LMDE/MGEN/Matmut}

法国医保的第二部分是补充险(Mutuelle),不是必须购买的,这部分由各大保险公司负责,病人可以自由选择保险公司。费用和报销比例视你自由选择的套餐而定。对于留学生,学校一般会为大家推荐购买学生医疗保险。

在法国看牙医或配眼镜,如果没有投保补充医疗保险的话,普通社会医疗保险机构的报销水平则会较差。

\subsubsection{CSS补充险}
对于低收入人群(如学生)可以申请CSS补充险,可参考\href{https://www.xiaohongshu.com/explore/6244b05d00000000210386be?note_flow_source=wechat}{链接}。

\subsubsection{住房保险Habitation} 
房屋保险是租房必须的手续之一。和房东或中介签署房屋合同前,明确是否需要你另外购买一份房屋保险。

如需购买,可在办理法国银行卡的同时在开卡银行咨询购买房屋保险的业务,或去专业的商业保险公司购买。

常见的商业保险公司有MAAF/MAE/Groupama等。

\subsection{住房补助CAF}
	
法国的房补政策是一种法国特有的社会福利,面向所有居住在法国的特定人群(包含学生)。强烈建议同学们抵达法国并获得房屋合同后尽快办理 CAF以便尽早享受房补。

按照规定,CAF自开户提交申请之日算起,且抵达法国的第一个月和彻底离开法国前的最后一个月是没有房补的。比如:

1). 8月份入境法国,房屋合同从8月或更早开始,如果在8月底前建立CAF账户并申请房补,第一笔房补就是从9月开始算的。 

2). 8月份入境法国,房屋合同从9月开始,即使在9月份建立CAF账户并申请房补,也只能从10月开始拿房补,因为9月是住房合同生效的第一个月。 

3). 8月份入境法国,房屋合同从8月或更早开始,但一直拖到10月份才建立CAF账户并申请房补,第一笔房补也只能从10月开始算。

虽然CAF的材料中要求提交银行卡RIB以及OFII证明,但这些材料是\textbf{可以之后补充}的,补材料的递交时间不影响第一笔房补的发放时间。入境时间、住房合同开始时间、建立CAF账户并申请房补的时间这三个因素决定了第一笔房补的起算月份。

CAF参考:\href{https://www.ecentime.com/article/demande-caf}{ECENTIME: CAF申请难到哭,详细到感动的申请指南来啦!}
 
\subsection{水电账户}

初到法国的小伙伴们在搬进住所之后可能会面临需要新办理用电账户的问题。法国比较常见的供电公司包括:EDF、Direct Energie、ENGIE等。

电账户可选择自行开户或从房东/上任房客那里过户,办理需要提供以下材料:有效的居留证件(带有学生签证的护照)、银行RIB、住房合同、上一任住户的信息和电表读数、电表设备号码。其中后面两项可咨询房东和供电公司。

电账户参考:
\begin{itemize}
    \item \href{https://www.ecentime.com/article/edf}{ECENTIME: 法国电力公司EDF办理保姆级教程来啦,手把手教你开户!}
    
    \item \href{https://www.ecentime.com/article/guide-electricite-comparaison-prix-bas}{ECENTIME: 开电指南,货比三家,办理迅速又省钱}
    
    \item \href{https://www.ecentime.com/article/edf-box-internet}{ECENTIME: 法国水电网开户办理流程!}
\end{itemize}

\subsection{中国大使馆教育处报到}

对于今后有回国就业规划的同学们,我们在海外的同学们需要注意一项关于学历学位认证的手续。首先,大家需在抵法后三个月内完成“留学人员登记”,随后在回国前要准备《留学回国人员证明》,最后,回国后要申请学位学历认证。由于使馆教育处无法联系到每一个来法人员,因此同学们需要自发完成登记备案工作,这样即使在今后出现人员失联、紧急案件等都能够凭备案信息联系家属,有备无患。具体登记通道为:\href{http://www.education-ambchine.org/}{http://www.education-ambchine.org/}。强烈建议大家在完成后牢记账号密码,因为到回国的时候可能由于时间久远加上手机号邮箱等的更新导致找回密码困难。

\subsection{报税}
在法国,每年四月是法国的报税月。虽然学生没有固定收入,仍然建议大家主动申报收入,因为报税年数越多,对于今后转换工作签证等越有帮助。另外报税也是证明自己是低收入或者无收入群体的必要步骤,有助于同学们申请CSS补充保险和\href{https://www.solidaritetransport.fr/}{Carte de Transport Solidaire}(一种福利性质的交通卡,每月18欧)。对于刚来法国的小伙伴,第一次报税需要在法国政府税务网站上开户,可在官网在线开户,或者去当地的税务局领取报税表格。

税务网址:\href{https://www.impots.gouv.fr/portail/}{https://www.impots.gouv.fr/portail/} 

在线开户需要填写的表格:
\href{https://www.impots.gouv.fr/portail/contacts?778}{https://www.impots.gouv.fr/portail/contacts?778}

报税参考:
\href{https://www.ecentime.com/article/ecentime-impots-france}{ECENTIME: 法国第一次报税}
