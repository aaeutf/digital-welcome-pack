% 法国及校友会简介
\section{法国及校友会简介}
\subsection{历史及政治}

法兰西共和国起始于1789年法国大革命,法国的国庆日7月14日即攻占巴士底狱的日子。从此开始,法国经历了持续百年波澜壮阔的革命。

法兰西第一共和国的历史血雨腥风,大家耳熟能详的吉伦特派、雅各宾派、罗伯斯庇尔、热月政变等历史人物和事件,就是发生在这个时期。

法兰西第一共和国遭到了拿破仑的复辟,1804年拿破仑在创建了法兰西第一帝国。我们在卢浮宫里可以看到著名的《拿破仑一世加冕大典》油画,记录的就是巴黎圣母院里拿破仑加冕的场景。

此后法兰西经历了一系列的政变:1815年拿破仑在滑铁卢战败后波旁王朝复辟,波旁王朝又在1830年被七月革命推翻,建立了法兰西第二共和国。1851年拿破仑的侄子又重演叔叔的历史,推翻了第二共和国成立了法兰西第二帝国。第二帝国在1870年被推翻又成立了第三共和国,一直持续到二战的时候法国被德国占领。

二战中德国占领了巴黎及法国北部,并在南部成立了以维希(Vichy)为首都的伪政权(即维希法国)。戴高乐将军在沦陷的法国组织了抵抗运动(La Résistance),并在二战胜利后法国成立了第四共和国。今天的法兰西第五共和国,则是1958年经历了阿尔及利亚独立战争中的政变后由戴高乐建立的。

法兰西第五共和国是现代民主政体,其核心设计和原则来源于法国18世纪哲学家孟德斯鸠提出的三权分立,即立法、行政和司法三种国家权利互相制衡的思想。法国的立法机关由国民议会(Assemblée Nationale)及参议院(Sénat)两院组成,政府由总统(Présidant)和总理(Premier Ministre)组阁并领导,而司法系统主要由各种法院(Tribunal)组成。法国每一位公民都拥有直接选举国家总统、国民议会议员、市长、区长等权利。

法兰西共和国的价值观暨座右铭是“自由,平等,博爱(Liberté, Égalité et Fraternité)”,这三个词会出现在法国的很多建筑和标记中。法国宪法保证公民的人权和自由:包括言论自由、信仰自由、著作和出版自由、集会及结社自由等。可以毫不夸张的说,说法国的宪法经历了两个多世纪的发展和改良,已经成为了西方民主政治的典范。而这部宪法的核心价值和原则,对自由、民主和人权的追求,也成为了法兰西文化的一个重要组成部分。

今天的法国总统马克龙(Emmanuel Macron)是“一起为了共和国”(Ensemble pour la République)党(中间偏右派)的代表。值得注意的是,“一起为了共和国”在2024年6月初的欧洲议会选举中落败于极右翼政党“国民联盟”(Rassemblement national),随后马克龙宣布解散国民议会,并于2024年7月7日完成了新一轮议员选举。目前左派新人民阵线(Nouveau Front Populaire)获得国民议会中的最多议席,但无任何政党或联盟取得绝对多数。由于法国的宪法规定政府的组成要与国民议会的组成一致,所以在2025年下半年之前法国政府的组建和运作会变得更加复杂。

下一次总统选举会是在2027年。

\subsection{风俗及传统}
法国人喜爱出门社交,比如在酒吧聊天、去餐馆吃饭、一同参观展览或在家里待客。法国较中国更喜爱轻礼,登门与会或生日婚礼可携带例如花束、红酒或者巧克力等伴手礼。

关于宗教:目前法国社会盛行的主流宗教有基督教(天主教、新教和东正教)、伊斯兰教和犹太教。自 1905 年起,法国实行世俗化政策,严格执行政教分离政策,即对所有宗教保持中立,不存在国家官方宗教。此项政策确立了法国宗教多元化的架构,在近些年受到一些挑战和质疑。

法国现在还保留夏令时和冬令时,夏令时期间和中国时差6小时,冬令时期间和中国时差7小时。每年3月最后一个周日从冬令时改成夏令时,调前1小时。每年10最后一个周日从夏令时改成冬令时,调后1小时。

法国传统节日(法定假日):

• 元旦,Nouvel An,1月1日

• 复活节,Pâque,一般在3月底到4月

• 劳动节,Fête du Travail,5月1日

• 二战停战日,Armistice de la Seconde Guerre Mondiale,5月8日

• 耶稣升天节,Ascension,复活节过40天,一般在5月中旬

• 圣灵降临节,Pentecôte,复活节后第七个周日,一般在五月下旬到6月中旬。

• 国庆节,Fête Nationale,7月14日

• 圣母升天节,Assomption,8月15日

• 诸圣瞻礼节,Toussaint,11月1日

• 一战停战日,Armistice de la Première Guerre Mondiale,11月11日

• 圣诞节,Noël,12月15日

法国城市介绍:\href{https://www.ecentime.com/article/-france-geographie}{ECENTIME: 法国大区城市介绍|最值得去的城市有哪些,快来看看你要去哪个迷人的地方?} 

\subsection{清华大学法国校友会(AAEUTF)}

清华大学法国校友会 AAEUTF(Association des Anciens Elèves de l’Université Tsinghua en France),是由旅法工作、交流和留学的清华大学校友自愿组成的联合性、非营利性的群众组织,2009年9月18日注册,2009年10月7日正式成立于法国巴黎。协会主要宗旨是服务校友,服务母校,促进中法科技文化交流。

我们的宗旨是:服务广大旅居法国和留学法国的校友,促进校友间的情谊,加强校友与母校之间的联系和团结,继承发扬“自强不息,厚德载物”的校训精神和“严谨、勤奋、求实、创新”的优良传统,为母校的发展和成为世界一流水平的大学做贡献。经过11年多的锻炼成长,清华大学法国校友会已经变成一个成熟的、有经验、有影响力的组织。随着更多的校友来法深造发展,法国清华校友会的队伍还在不断壮大,影响也在继续提升,逐渐成为了中国留法学人最喜爱的校友会之一,清华人的精神在法兰西土地之上得到了发扬光大。

清华大学法国校友会理事会是校友会的决策机构,现任会长是康嘉,秘书长为刘大伟。此外,校友会在2022年开始还成立了顾问团,对理事会进行建议及沟通。

校友会相关信息:
\begin{itemize}
    \item 校友会官网:\href{https://www.tsinghua-france.org}{https://www.tsinghua-france.org}
    \item 校友会会员注册(学生免费):\href{https://www.tsinghua-france.org/about/registration/}{https://www.tsinghua-france.org/about/registration/}
    \item 理事会电子邮箱:secretariat@tsinghua-france.org 
    \item 校友会微信公众号:AAEUTF(二维码见封面页)
    \item 讨论群:我们在微信和Telegram上都有群组,你可以通过联系理事会加入。
\end{itemize}

\subsection{中国驻法国大使馆与学联}

中国驻法国大使馆于1964 年1 月 7 日中法建交后成立与法国巴黎,现任大使为卢沙野先生,使馆现地址为20, Rue Monsieur, 75007 Paris,使馆官网为\href{http://www.amb-chine.fr/chn/}{http://www.amb-chine.fr/chn/}。另在法国本土与海外省设5领馆:斯特拉斯堡、马赛、里昂、圣丹尼、帕皮提。留学在外,建议同学们时刻关注使馆动态发布。

另外,全法中国学者学生联合会(UCECF简称“全法学联”)是旅法中国留学人员自我管理,自我服务的群众性组织,是中国驻法使馆教育处联系广大留学人员的重要纽带。法学联在全法国各留学人员密集的城市均设有分学联, 覆盖了整个法国版图 (包括科西嘉岛)。全法学联会更新一版留学生新生手册,可以从微信公众号下载。全法学联官网:\href{www.ucecf.fr}{www.ucecf.fr},微信公众号:全法学联。
