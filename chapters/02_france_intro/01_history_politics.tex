\subsection{历史及政治}

法兰西共和国起始于1789年法国大革命,法国的国庆日7月14日即攻占巴士底狱的日子。从此开始,法国经历了持续百年波澜壮阔的革命。

法兰西第一共和国的历史血雨腥风,大家耳熟能详的吉伦特派、雅各宾派、罗伯斯庇尔、热月政变等历史人物和事件,就是发生在这个时期。

法兰西第一共和国遭到了拿破仑的复辟,1804年拿破仑在创建了法兰西第一帝国。我们在卢浮宫里可以看到著名的《拿破仑一世加冕大典》油画,记录的就是巴黎圣母院里拿破仑加冕的场景。

此后法兰西经历了一系列的政变:1815年拿破仑在滑铁卢战败后波旁王朝复辟,波旁王朝又在1830年被七月革命推翻,建立了法兰西第二共和国。1851年拿破仑的侄子又重演叔叔的历史,推翻了第二共和国成立了法兰西第二帝国。第二帝国在1870年被推翻又成立了第三共和国,一直持续到二战的时候法国被德国占领。

二战中德国占领了巴黎及法国北部,并在南部成立了以维希(Vichy)为首都的伪政权(即维希法国)。戴高乐将军在沦陷的法国组织了抵抗运动(La Résistance),并在二战胜利后法国成立了第四共和国。今天的法兰西第五共和国,则是1958年经历了阿尔及利亚独立战争中的政变后由戴高乐建立的。

法兰西第五共和国是现代民主政体,其核心设计和原则来源于法国18世纪哲学家孟德斯鸠提出的三权分立,即立法、行政和司法三种国家权利互相制衡的思想。法国的立法机关由国民议会(Assemblée Nationale)及参议院(Sénat)两院组成,政府由总统(Présidant)和总理(Premier Ministre)组阁并领导,而司法系统主要由各种法院(Tribunal)组成。法国每一位公民都拥有直接选举国家总统、国民议会议员、市长、区长等权利。

法兰西共和国的价值观暨座右铭是“自由,平等,博爱(Liberté, Égalité et Fraternité)”,这三个词会出现在法国的很多建筑和标记中。法国宪法保证公民的人权和自由:包括言论自由、信仰自由、著作和出版自由、集会及结社自由等。可以毫不夸张的说,说法国的宪法经历了两个多世纪的发展和改良,已经成为了西方民主政治的典范。而这部宪法的核心价值和原则,对自由、民主和人权的追求,也成为了法兰西文化的一个重要组成部分。

今天的法国总统马克龙(Emmanuel Macron)是“一起为了共和国”(Ensemble pour la République)党(中间偏右派)的代表。值得注意的是,“一起为了共和国”在2024年6月初的欧洲议会选举中落败于极右翼政党“国民联盟”(Rassemblement national),随后马克龙宣布解散国民议会,并于2024年7月7日完成了新一轮议员选举。目前左派新人民阵线(Nouveau Front Populaire)获得国民议会中的最多议席,但无任何政党或联盟取得绝对多数。由于法国的宪法规定政府的组成要与国民议会的组成一致,所以在2025年下半年之前法国政府的组建和运作会变得更加复杂。

下一次总统选举会是在2027年。
