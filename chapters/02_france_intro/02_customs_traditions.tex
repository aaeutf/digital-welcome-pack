\subsection{风俗及传统}
法国人喜爱出门社交,比如在酒吧聊天、去餐馆吃饭、一同参观展览或在家里待客。法国较中国更喜爱轻礼,登门与会或生日婚礼可携带例如花束、红酒或者巧克力等伴手礼。

关于宗教:目前法国社会盛行的主流宗教有基督教(天主教、新教和东正教)、伊斯兰教和犹太教。自 1905 年起,法国实行世俗化政策,严格执行政教分离政策,即对所有宗教保持中立,不存在国家官方宗教。此项政策确立了法国宗教多元化的架构,在近些年受到一些挑战和质疑。

法国现在还保留夏令时和冬令时,夏令时期间和中国时差6小时,冬令时期间和中国时差7小时。每年3月最后一个周日从冬令时改成夏令时,调前1小时。每年10最后一个周日从夏令时改成冬令时,调后1小时。

法国传统节日(法定假日):

• 元旦,Nouvel An,1月1日

• 复活节,Pâque,一般在3月底到4月

• 劳动节,Fête du Travail,5月1日

• 二战停战日,Armistice de la Seconde Guerre Mondiale,5月8日

• 耶稣升天节,Ascension,复活节过40天,一般在5月中旬

• 圣灵降临节,Pentecôte,复活节后第七个周日,一般在五月下旬到6月中旬。

• 国庆节,Fête Nationale,7月14日

• 圣母升天节,Assomption,8月15日

• 诸圣瞻礼节,Toussaint,11月1日

• 一战停战日,Armistice de la Première Guerre Mondiale,11月11日

• 圣诞节,Noël,12月15日

法国城市介绍:\href{https://www.ecentime.com/article/-france-geographie}{ECENTIME: 法国大区城市介绍|最值得去的城市有哪些,快来看看你要去哪个迷人的地方?} 
