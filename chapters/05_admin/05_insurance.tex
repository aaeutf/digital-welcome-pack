\subsection{保险}
\subsubsection{社会医疗保险}

法国的医保体制由两部分组成:基本医疗保险以及补充险。基本医疗保险由法国医保部门Assurance maladies负责任何在法国拥有合法拘留权的人都包括在内。费用视你的收入而定。基本险一般报销各类诊疗费用的60\%-70\%。余下部分的报销就由补充险赔付。

法国学生医疗保险制度将进行改革,新推出的CVEC(La Contribution Vie Etudiante et de Camps)取代了原来的学生社会保险(La Sécurité Sociale Etudiante), 以90欧/年获得医保卡(Carte Vitale), 享受医疗保障包括门诊、就医、买药等最高70\%的报销额度。购买可通过\href{https://www.messervices.etudiant.gouv.fr}{网站}进行。付款成功后下载保留注册证明,在去学校注册时需要提供。购买成功后需要要在社保网站注册生效。

法国行政效率较为缓慢,可以通过电话沟通加快获得Carte Vitale和激活ameli账户的速度。\href{https://www.ameli.fr/paris/assure/english-pages}{电话参考}:09 74 75 36 46。

所需材料:学校注册证明、护照、学生签证、出生公证、居留证/OFII、银行账户信息RIB。

社保参考:\href{https://www.ecentime.com/article/assurance-maladie}{ECENTIME: 法国医保、学生医保不会办理?保姆级申请攻略你值得拥有!}
 
\subsubsection{辅助保险Mutuelle – LMDE/MGEN/Matmut}

法国医保的第二部分是补充险(Mutuelle),不是必须购买的,这部分由各大保险公司负责,病人可以自由选择保险公司。费用和报销比例视你自由选择的套餐而定。对于留学生,学校一般会为大家推荐购买学生医疗保险。

在法国看牙医或配眼镜,如果没有投保补充医疗保险的话,普通社会医疗保险机构的报销水平则会较差。

\subsubsection{CSS补充险}
对于低收入人群(如学生)可以申请CSS补充险,可参考\href{https://www.xiaohongshu.com/explore/6244b05d00000000210386be?note_flow_source=wechat}{链接}。

\subsubsection{住房保险Habitation} 
房屋保险是租房必须的手续之一。和房东或中介签署房屋合同前,明确是否需要你另外购买一份房屋保险。

如需购买,可在办理法国银行卡的同时在开卡银行咨询购买房屋保险的业务,或去专业的商业保险公司购买。

常见的商业保险公司有MAAF/MAE/Groupama等。
