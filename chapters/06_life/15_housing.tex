% 住房租赁与找房攻略
\section{住房租赁与找房攻略}

\subsection{常见找房渠道}
\begin{itemize}
    \item \textbf{学校宿舍/学生公寓}:如CROUS、各高校自有公寓,适合新生,安全有保障,申请需提前。
    \item \textbf{租房平台}:如Le Bon Coin、SeLoger、PAP、Studapart、Appartager(合租)、Facebook租房群等。
    \item \textbf{中介公司}:可提供专业服务,需支付中介费。
    \item \textbf{校友会/微信群}:通过校友、朋友介绍,信息更可靠。
\end{itemize}

\subsection{租房流程简述}
\begin{enumerate}
    \item 明确预算、位置、房型需求。
    \item 在线浏览房源,联系房东/中介预约看房。
    \item 看房时注意房屋状况、周边环境、交通便利性。
    \item 提交材料(身份证明、学生证明、担保人、收入证明等)。
    \item 签署租房合同(bail),仔细阅读条款,保留副本。
    \item 入住时与房东共同完成入住检查(état des lieux),记录房屋现状。
    \item 缴纳押金(dépôt de garantie)和首期房租,办理水电网等开户。
\end{enumerate}

\subsection{注意事项}
\begin{itemize}
    \item 法国租房合同通常为一年(可续),提前解约需遵守通知期(préavis)。
    \item 谨防诈骗,切勿提前转账押金,务必实地看房。
    \item 保留所有沟通和付款凭证。
    \item 入住后如发现问题,及时与房东沟通并留存书面记录。
    \item 可申请房租补贴(CAF),需准备相关材料。
\end{itemize}

\subsection{防范找房诈骗}
\begin{itemize}
    \item 切勿在未签合同、未看房前提前转账押金或房租。
    \item 谨防低价诱惑和紧迫催促,务必实地看房或请可信朋友代看。
    \item 核实房东/中介身份,可要求出示身份证明、房产证明。
    \item 保留所有沟通和付款凭证,遇可疑情况及时向校友会、学校或法律咨询中心求助。
    \item 常见手法包括假冒房东、中介,伪造证件,要求通过不正规渠道付款等。
\end{itemize}

\subsection{常用法语词汇}
\begin{itemize}
    \item logement(住房)、bail(租房合同)、caution/dépôt de garantie(押金)、quittance de loyer(租金收据)、état des lieux(房屋状况检查)、colocation(合租)、propriétaire(房东)、locataire(租客)、charges(杂费)。
\end{itemize}
