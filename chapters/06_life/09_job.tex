\subsection{打工与兼职须知}

法国留学生可合法兼职,每年最多工作964小时(约每周17-20小时),超时需雇主申请临时工作许可证。常见兼职包括餐饮、超市、家教、翻译、助教、派发传单等。博士生可申请助教岗位(TD/TP)。

\subsubsection{如何找兼职}
\begin{itemize}
    \item 关注学校、CROUS、校友会、微信群、Facebook群等信息渠道。
    \item 浏览求职网站(如L'Étudiant、Indeed、Pôle Emploi)、超市/商店公告栏。
    \item 咨询中介公司(Agence d’intérim)获取短期岗位。
\end{itemize}

\subsubsection{签订劳动合同}
\begin{itemize}
    \item 劳动合同(Contrat de travail)须注明岗位、工资、工作时间、期限等,签字前务必仔细阅读并保留副本。
    \item 工资须通过银行转账发放,注意核对工资单(fiche de paie)。
\end{itemize}

\subsubsection{注意事项}
\begin{itemize}
    \item 求职时警惕虚假信息和诈骗,勿随意泄露个人信息或提前转账。
    \item 合理安排学业与工作,优先保证学习进度。
    \item 如遇劳动纠纷,可向学校、校友会或当地法律援助机构寻求帮助。
\end{itemize}
