\subsection{心理健康与适应}

初到法国,可能会遇到文化冲击、孤独、学业压力等问题。法国各高校一般设有心理咨询中心(Service de santé universitaire, SSU),可免费预约心理咨询。常用心理健康热线包括Fil Santé Jeunes(0800 235 236,免费,适合青少年),SOS Amitié(见安全章节)。如有需要,建议及时寻求帮助,保持身心健康。此外,遇到心理压力时,也可以通过校友会(如AAEUTF等)提供的微信群、公众号、线下活动等多种途径联系校友,寻求经验分享、情感支持或实际帮助。校友会成员通常乐于倾听和协助,建议积极利用这些资源。

\subsubsection{文化适应与心理调适}
初到法国,常见的心理变化包括:新鲜期、文化冲击期、调整适应期和融入期。遇到情绪波动、孤独、焦虑等属正常现象。建议保持规律作息、适度运动、主动与同学朋友交流,积极参与校友会及社团活动,逐步建立支持网络。

\subsubsection{法国心理健康医疗体系简介}
法国医疗体系对心理健康较为重视。可通过家庭医生(médecin traitant)转诊心理医生(psychologue)或精神科医生(psychiatre)。部分心理咨询可报销,18-25岁学生可通过“MonPsy”计划享受免费心理咨询(详见 https://www.ameli.fr/)。

\subsubsection{线上心理健康资源与App}
\begin{itemize}
    \item \textbf{Fil Santé Jeunes}(https://www.filsantejeunes.com/):青少年心理健康平台,含匿名咨询与丰富资料。
    \item \textbf{Nightline France}(https://www.nightline.fr/):专为学生提供的夜间心理支持热线和在线聊天服务。
    \item \textbf{App推荐:}"Calm"、"Headspace"(冥想)、"Mon Sherpa"(法语心理自助)等。
\end{itemize}

\subsubsection{紧急心理危机应对}
如有严重心理危机(如自伤、自杀念头等),请立即联系:
\begin{itemize}
    \item 紧急电话:15(医疗急救)、112(欧盟通用)、3114(法国心理危机干预热线,24小时)
    \item 就近医院急诊室(Urgences)
    \item 校医院或心理健康中心
\end{itemize}

\subsubsection{正视心理健康}
心理健康与身体健康同等重要。遇到困扰时,主动寻求帮助是成熟和负责任的表现。法国社会对心理健康较为包容,建议消除羞耻感,善用各类资源。
