\subsection{交通}

\subsubsection{自行车}
在法国,全新的自行车大概平时价格在200欧左右,可在自行车商店或法国运动品商店迪卡侬Decathlon购买。二手自行车可以通过法国二手商品网站\href{https://www.leboncoin.fr/}{Leboncoin}查找购买。

另外,想临时骑车代步的小伙伴可查询所在城市是否有公共自行车,如巴黎的\href{http://www.velib.paris/}{Vélib},里尔的\href{https://www.lillemetropole.fr/votre-quotidien/se-deplacer/se-deplacer-velo}{V’lille} 

\subsubsection{市内交通}
于巴黎戴高乐机场,抵达后有不少中文指示牌,且有信息服务台和工作人员提供指引。前往市区可乘坐快轨(约 14 欧)、巴士(约12欧)或出租车(50-60欧固定费用)。

关于巴黎市内交通:有公共交通例如地铁(Metro)/快轨(RER)/小火车(Train)/电车(Tram)等,有共享交通例如巴黎自行车、脚踏车等的租赁。由于城市对车辆限制逐渐升级,道路限行/限速使得市内公共交通更受欢迎。必要时,可预约出租车G7、网约车Uber、Bolt等。

\subsubsection{SNCF/青年卡}

法国国家铁路公司SNCF是小伙伴们在法国出远门最依赖的交通方式,有高速铁路(TGV)贯穿全国、方便快捷省际列车TER直达相邻城市、低价的非高峰高铁Ouigo专列。而SNCF的票价浮动大不稳定,各式优惠卡、会员卡则能帮大家省钱:有适合频繁乘坐高铁的月卡TGV Max(27岁及以下)、年卡Forfait Annuel TGV,还有适合年轻人的青年卡(年卡)Carte Avantage Jeune(26岁及以下)等等,详情请见\href{https://www.ecentime.com/article/avantage-carte-sncf}{Link}。

交通参考:
\begin{itemize}
    \item \href{https://www.ecentime.com/article/aeroport-CDG%202020}{ECENTIME: 法国戴高乐机场CDG完全指南}
    \item \href{https://www.ecentime.com/article/transport-paris}{ECENTIME: 巴黎交通防丢指南}
    \item \href{https://www.ecentime.com/article/reduction-transport-etudiant}{ECENTIME: 学生公交卡办理攻略:巴黎、斯堡\&里昂的学生们看过来啦!}
    \item \href{https://www.ecentime.com/article/ecentime-sncf-reduction}{ECENTIME: SNCF最新优惠卡政策对比}
\end{itemize}

\subsubsection{Carte de Transport Solidaire}
通过报税证明自己是低收入或者无收入群体,并申请成功申请CSS后,同学们可以申请\href{https://www.solidaritetransport.fr/}{Carte de Transport Solidaire}(一种福利性质的交通卡,每月18欧)。

\subsubsection{自驾}
持学⽣长居卡的⼩伙伴,可直接使用经认证翻译的中国驾照。但是由于中法间交规有别,建议有驾驶经验的⼩伙伴出发前⼀定要确保驾照可用并注意遵守法国交规,安全驾驶!
