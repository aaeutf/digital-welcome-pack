\subsection{法律常识}

\begin{itemize}
    \item 租房、劳动合同等重要文件请务必仔细阅读并保留副本,遇到纠纷可咨询学校法律援助或当地法律咨询中心(Maison de Justice et du Droit)。
    \item 法国警察执法时应积极配合,随身携带身份证件复印件。
    \item 购买、持有、使用毒品在法国属于违法行为,后果严重。
    \item 交通规则严格,酒驾、闯红灯等违法行为处罚严厉。
    \item 网络诈骗、电话诈骗较多,涉及金钱转账务必核实对方身份。
    \item 法国法律严禁种族歧视、仇恨言论和歧视性行为,若遇到相关情况可向学校、校友会或相关机构(如Défenseur des droits)举报,维护自身合法权益。
\end{itemize}

\subsubsection{法律援助与求助渠道}
\begin{itemize}
    \item \textbf{学校法律援助:} 大部分高校为学生提供免费法律咨询服务,遇到纠纷可优先联系学校学生事务处或国际学生办公室。
    \item \textbf{Maison de Justice et du Droit:} 法国各地设有法律与权利之家,提供免费法律咨询和调解服务,可通过市政厅网站查询最近的地址。
    \item \textbf{Défenseur des droits:} 法国国家权利保护机构,受理歧视、警察执法不当等投诉,官网 https://www.defenseurdesdroits.fr/。
    \item \textbf{免费法律咨询热线:} 如 3039(法国本地拨打),可获得初步法律建议。
    \item \textbf{中国驻法使领馆:} 在遇到重大法律问题或人身安全威胁时,可联系中国驻法使领馆寻求协助。
\end{itemize}

\subsubsection{留学生常见法律问题及应对}
\begin{itemize}
    \item \textbf{签证与居留:} 持合法签证和居留许可,及时办理延续手续,避免非法滞留。
    \item \textbf{劳动权益:} 兼职需签订劳动合同,了解最低工资标准(SMIC),如遇拖欠工资、无故解雇等可向劳动监察部门(Inspection du travail)投诉。
    \item \textbf{租房纠纷:} 建议签订正式租房合同,保留押金、房屋状况交接单等证据,遇纠纷可向ADIL(住房信息机构)或法律援助机构咨询。
    \item \textbf{歧视与骚扰:} 法国法律严禁歧视和骚扰,遇到相关情况可向学校、校友会或Défenseur des droits举报。
    \item \textbf{交通事故:} 发生交通事故应及时报警(拨打17),并拍照取证,必要时联系保险公司。
\end{itemize}

\subsubsection{证据保存与报警流程}
\begin{itemize}
    \item 保留所有相关文件、合同、聊天记录、照片等证据。
    \item 报警时需说明事件经过、时间、地点、涉事人员等,尽量提供书面材料。
    \item 紧急情况下可拨打17(警察)、112(欧盟通用紧急电话)。
\end{itemize}

\subsubsection{常用法律资源与App}
\begin{itemize}
    \item Service-Public.fr(https://www.service-public.fr/):法国官方公共服务门户,涵盖各类法律信息。
    \item ADIL(https://www.anil.org/):住房法律咨询。
    \item 法国司法部(https://www.justice.fr/):法律程序、援助信息。
    \item App推荐:"Justice.fr"、"Service-Public.fr"等。
\end{itemize}

\subsubsection{校友会与使馆支持}
\begin{itemize}
    \item 校友会可协助提供法律咨询资源、经验分享和心理支持。
    \item 遇到重大法律问题或紧急情况,可联系中国驻法使领馆,获取领事保护和协助。
\end{itemize}
