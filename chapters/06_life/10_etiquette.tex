\subsection{日常礼仪}

\subsubsection{积极融入当地文化}
踊跃参与课外活动,尝试结交新朋友,积极参与讨论和研究,以开放的心态结实志同道合的朋友。当地人喜爱出门社交,比如在酒吧聊天、去餐馆吃饭、一同参观展览或在家里待客。法国较中国更喜爱轻礼,登门与会或生日婚礼可携带例如花束、红酒或者巧克力等伴手礼。

\subsubsection{尝试学习法语}
法语是法国的官方语言,许多年轻人也有良好的英文水平。然而法语的支配地位决定了学习法语是能够增加你在法国生活厚度、深度和舒适度的最好捷径。不论是学习工作,还是看病、交友,基本的法语沟通都会为你带来便利,使你更容易感受到法国生活的魅力。学习方法:首选学校的法语课,其次每个市政府都有语言中心提供帮助,还有很多社会组织、商业学校可以提供课程。

\subsubsection{尊重当地政治、文化与宗教,求同存异}
法国的个人社会观点比较强烈,善辩论,与人辩论需秉承客观,避免过多的感情介入。

\subsubsection{Dos}
见面要说你好Bonjour, 待人接物要说谢谢Merci, 与他人交流要习惯眼神注视。

\subsubsection{Don'ts}
走路不要拉横排,排队不要与他人距离过近(尤其切记任何身体触碰),遇到可疑人员不要过度明显地绕道或者露怯以免激起坏人临时起意的作恶,不可将任何包裹(行李、书包等)单独留在公共空间(去洗手间、付费等)。

文化差异参考:\href{https://www.ecentime.com/article/difference-culturelle}{ECENTIME: 法中文化差异大盘点,来法留学的你一定要知道这些!}
