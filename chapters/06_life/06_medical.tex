\subsection{医疗}

在法国,只要大家办理好了Carte Vitale医保卡,看病不难也不贵。

\subsubsection{全科医生}	
全科医生(Médecin Généraliste)提供基本的疾病诊疗和全面检查。在法国,每个享受医保的人需要申报一位家庭医生(Médecin Traitant),其一般情况下为全科医生,作为今后不管生什么病的一个主要联系人。家庭医生看病有问诊费,不过可以报销,且医生之后开的处方药也都是可以报销的。

首次选择和申报家庭医生需要下载后找到你所选择的医生共同签署一个申报表格Déclaration De Choix Du Médecin Traitant,寄给医保机构。但在选择你的家庭医生前,最好电话/邮件咨询一下,因为有一定数量的医生不再接受新患者来签订家庭医生。

在法国,就医需要预约。推荐一个全法通用的就医预约软件 Doctolib,既可以预约,还可以实现远程问诊。

\subsubsection{专科医生}
专科医生(Médecin Spécialiste)是专攻于某些领域的医生。当全科医生遇到无法解决的问题时会推荐一位专科医生。大部分专科医生需要全科医生的转诊,但是眼科医生和牙医一般不需要。

\subsubsection{校医}
对于在校生,可以选择校医。不同学校校医的报销情况有所不同,需要及时向学校咨询。

\subsubsection{药房}
如果得了一些小病,比如感冒、流鼻涕、喉咙痛、发热等等,大家觉得没有太大的必要去医院,就可以直接去家里附近的药店,药店的员工都是有专业资格证的,他们能够直接帮助解决这些小的病痛,而且能够及时推荐适合的药物。此外,药房还可以帮助注射疫苗(如HPV疫苗),但需要提前电话咨询。在法国,凡需凭医生处方出售的药物均由药房垄断经营。药房的绿色十字标志赫然醒目,每个街城市通常都有至少一个值班药房,以满足紧急需求。

\subsubsection{化验中心}
当医生开具处方需要你进行化验血检的话,可以前往化验机构(Laboratoire)。可以在谷歌地图搜索离家较近的化验中心,一般不需要预约,带好处方和医保卡前往即可。但如果不放心也可以选择约定时间。

\subsubsection{医院}
在法国医院分为公立医院、普通私立诊所和特约私立诊所,区别主要是在于看病的效率和报销的程度,当然看病的效率三种医院自然是从低到高,而报销的程度则是相反了。

公立医院里有开专门的急诊,如果是在晚上需要看诊可以直接去公立医院问诊,不过在公立医院问诊的话有一笔费用是需要自己处理并无法被报销的。如果大家在看急诊的过程中呼叫了救护车,这一笔费用也是需要从自己腰包里掏出来的哦,而且一般都是上百欧元了,所以呼叫救护车需谨慎。

大家如果真的遇到了紧急的问题,可以直接拨打紧急医疗救护处15进行咨询,拨打15的好处在于,他们知道你家附近所有的医疗状况,可以帮你推荐最佳的医院、诊所或者医生,而且在自己没有车的情况下他们也可以为你派车来你家接送,并且效率很高。

医疗参考:
\begin{itemize}
    \item \href{https://www.ecentime.com/article/ecentime-ameli-gratuit}{ECENTIME: 法国医保福利,免费体检?免费看牙?还有宫颈癌筛查项目!}
    \item \href{https://www.ecentime.com/article/comment-utiliser-doctolib-pendant-le-confinement}{ECENTIME: 宅家就医神器Doctolib}
    \item \href{https://www.dealmoon.fr/guide/1366}{ECENTIME: 法国常见病和就医常用句}
    \item \href{https://www.ecentime.com/article/2018-pharmacie-list}{ECENTIME: 急救药品指南:在法活着基本靠自己撑}
\end{itemize}
