% 生活
\section{生活}
生活参考:\href{https://www.ecentime.com/article/liste-app}{ECENTIME: 无缝衔接中法生活,你需要这些App!}


\subsection{超市}
	
法国超市品牌有Carrefour, Géant, Auchan, Franprix, Lidl, Casino, SuperU, Intermarché等,此外还有有机生活超市(法国的有机食品会标注“bio”)、市场。另外出于减少食物浪费和环境保护的考虑,一些商家(除超市外还有面包甜品店、餐馆等)会将当天无法售出但依旧新鲜的食物以较低价格上架到TooGoodToGo软件平台上,俗称“剩菜盲盒”。

实体中超品牌主要集中在大巴黎地区。巴黎有\href{https://www.tang-freres.fr/}{陈氏(Tang Frère)}、\href{https://paris-store.com/}{巴黎士多(Pairs Store)}、\href{https://www.instagram.com/chenmarketfr/?hl=en}{中国红(Chen Market)}、新今日、大中华等。韩超有\href{https://www.instagram.com/kmartfrance/?hl=en}{K-Mart}。

里尔地区:

\href{https://maps.app.goo.gl/arekUkxV9KnKrmaj8}{Asie Nord, 15-17 Rue Jules Guesde, 59000 Lille}

\href{https://maps.app.goo.gl/4teSNmTntTzK39y36}{Paris Store, 23 Rue du Collège, 59100 Roubaix}

线上中超软件包括\href{https://mywaysia.com/en}{方圆食里Waysia}(校友经营,可送至巴黎以外地区)、悟空送菜(巴黎地区)、打酱油(德国可外送法国)等,他们提供中国食材和24$\times$7外送服务。

超市参考:
\begin{itemize}
    \item \href{https://www.ecentime.com/article/supermarket}{ECENTIME: 超市知多少——法国生活必备的超市有哪些?}
    \item \href{https://www.ecentime.com/article/ecentime-supermarketfrance}{ECENTIME: 超市会员卡大测评}
\end{itemize}


\subsection{饮食}

\subsubsection{大学食堂}
大学食堂CROUS是法国政府补贴的学生食堂,价格低廉、餐食简单。全法国共有800多个大学食堂。法国的大学餐厅 (Restaurants universitaires) 供应的菜式大抵依前菜、主菜、点心等三四道菜配成,面包可随意取用,饮料须另外购买。

首先在你的Crous卡(类似中国大学的校园一卡通)或是学生证里充钱,校园内可能有可以冲卡的机器(Borne),直接用银行卡充值即可。或者使用Izly软件充值。食堂价格在各城市会有差异,举例来说,巴黎一家CROUS(3.30欧)的一餐包括6个点(Point),按照点数自助选取。一般一份主菜Plat在3-4个点;沙拉1个点;水果1个点;汤类1个点;甜点1-2个点。绝大多数Resto U只在午餐时间开放(11:30 - 14:30)。

\subsubsection{餐馆酒吧}
餐馆以法餐为主,主要分为星级餐馆(étoile),普通餐馆(Brasserie/Bouillon/Bistro等),简食餐厅(Crêperie /Bar à salade等)。另有各国料理(日韩、印度、泰国、意式、美式、北非及中东地区美食等等)。在法国有越来越多的中餐馆,从盒饭快餐厅(Traiteur Chinois)到与国内接轨的地方特色美食一应俱全。巴黎地区中餐餐馆选择众多,大部分可外送至巴黎。

里尔地区:\href{https://maps.app.goo.gl/ScHQq7jbV4Bug9JJ7}{Tsingtao烤鸭店 13 Rue Jules Guesde, 59000 Lille}

在法国,随处可见咖啡厅和酒馆,是绝大部分法国人十分青睐的与会友闲聊首选场所。同时也慢慢有越来越多的特色奶茶店与甜品店可供国人聚会消遣。

餐饮参考:
\begin{itemize}
    \item \href{https://www.ecentime.com/article/comment-profider-le-crous-en-france}{ECENTIME: 法国食堂Crous你还没好好利用起来么?}
    
    \item \href{https://www.ecentime.com/article/ecentime-bar-paris}{ECENTIME: 巴黎酒吧周末好去处}
\end{itemize}

\subsection{垃圾分类}

法国的垃圾需要严格分类才可以丢弃,随意丢弃垃圾可能会收到罚款。常见的垃圾分为:

\begin{itemize}
    \item 可回收垃圾(Déchets recyclables):包装盒、塑料袋、塑料瓶、金属等散装放入黄色的垃圾桶。玻璃瓶放入玻璃制品特定的垃圾桶。
    \item 普通垃圾:无法回收的家庭垃圾,放入密封垃圾袋丢到灰色或棕色垃圾桶。
    \item 有害垃圾:包括药品、电池、电器、灯管等有害垃圾需要丢弃到专门的回收地点,请查看居住城市的相关规定。
    \item 废旧家具:请查看居住城市的相关规定,一般情况下每个月或每个礼拜会有固定的时间统一回收。
    \item 装修垃圾:需要自行送到垃圾处理厂,请查看居住城市的相关规定。
\end{itemize}

\subsection{交通}

\subsubsection{自行车}
在法国,全新的自行车大概平时价格在200欧左右,可在自行车商店或法国运动品商店迪卡侬Decathlon购买。二手自行车可以通过法国二手商品网站\href{https://www.leboncoin.fr/}{Leboncoin}查找购买。

另外,想临时骑车代步的小伙伴可查询所在城市是否有公共自行车,如巴黎的\href{http://www.velib.paris/}{Vélib},里尔的\href{https://www.lillemetropole.fr/votre-quotidien/se-deplacer/se-deplacer-velo}{V’lille} 

\subsubsection{市内交通}
于巴黎戴高乐机场,抵达后有不少中文指示牌,且有信息服务台和工作人员提供指引。前往市区可乘坐快轨(约 14 欧)、巴士(约12欧)或出租车(50-60欧固定费用)。

关于巴黎市内交通:有公共交通例如地铁(Metro)/快轨(RER)/小火车(Train)/电车(Tram)等,有共享交通例如巴黎自行车、脚踏车等的租赁。由于城市对车辆限制逐渐升级,道路限行/限速使得市内公共交通更受欢迎。必要时,可预约出租车G7、网约车Uber、Bolt等。

\subsubsection{SNCF/青年卡}

法国国家铁路公司SNCF是小伙伴们在法国出远门最依赖的交通方式,有高速铁路(TGV)贯穿全国、方便快捷省际列车TER直达相邻城市、低价的非高峰高铁Ouigo专列。而SNCF的票价浮动大不稳定,各式优惠卡、会员卡则能帮大家省钱:有适合频繁乘坐高铁的月卡TGV Max(27岁及以下)、年卡Forfait Annuel TGV,还有适合年轻人的青年卡(年卡)Carte Avantage Jeune(26岁及以下)等等,详情请见\href{https://www.ecentime.com/article/avantage-carte-sncf}{Link}。

交通参考:
\begin{itemize}
    \item \href{https://www.ecentime.com/article/aeroport-CDG%202020}{ECENTIME: 法国戴高乐机场CDG完全指南}
    \item \href{https://www.ecentime.com/article/transport-paris}{ECENTIME: 巴黎交通防丢指南}
    \item \href{https://www.ecentime.com/article/reduction-transport-etudiant}{ECENTIME: 学生公交卡办理攻略:巴黎、斯堡\&里昂的学生们看过来啦!}
    \item \href{https://www.ecentime.com/article/ecentime-sncf-reduction}{ECENTIME: SNCF最新优惠卡政策对比}
\end{itemize}

\subsubsection{Carte de Transport Solidaire}
通过报税证明自己是低收入或者无收入群体,并申请成功申请CSS后,同学们可以申请\href{https://www.solidaritetransport.fr/}{Carte de Transport Solidaire}(一种福利性质的交通卡,每月18欧)。

\subsubsection{自驾}
持学⽣长居卡的⼩伙伴,可直接使用经认证翻译的中国驾照。但是由于中法间交规有别,建议有驾驶经验的⼩伙伴出发前⼀定要确保驾照可用并注意遵守法国交规,安全驾驶!

\subsection{购物}
	
巴黎地区的商场:老佛爷(Galerie Lafayette), 巴黎春天(Printemps)、BHV、乐蓬马歇(Le Bon Marche)、新开的Sanmaritaine, 巴黎中心 Chatelet 地铁站的Westfield,拉德芳斯商务区的 Quatres Temps,巴黎郊区的打折村 Vallée Village等。另有多条购物街区:6区的Rue de four, 4区的玛黑区,11区外玛黑区等。

里尔:巴黎春天(Printemps)、EuraLille

法国的打折季每年两季,其中夏季打折季每年六月底七月初开始,冬季打折季每年一月开始。

网购参考:
\begin{itemize}
    \item \href{https://www.ecentime.com/article/ecommerce-fr-top15}{ECENTIME: 逛啥淘宝!法国网购电商TOP15出炉!}
    \item \href{https://www.ecentime.com/article/online-shopping-delivery}{ECENTIME: 法国网购邮寄方式汇总}
    \item \href{https://www.ecentime.com/article/amazon-fr-skill-1}{ECENTIME: 亚马逊Prime法国攻略及隐藏福利}
\end{itemize}

出于环保和节约的考虑,法国的二手文化较为流行。常见的网站和平台有\href{https://www.leboncoin.fr/}{leboncoin}、\href{https://www.vinted.fr/}{Vinted}(二手衣物)、\href{https://www.momox-shop.fr/}{momox}(二手书籍)等。


\subsection{医疗}

在法国,只要大家办理好了Carte Vitale医保卡,看病不难也不贵。

\subsubsection{全科医生}	
全科医生(Médecin Généraliste)提供基本的疾病诊疗和全面检查。在法国,每个享受医保的人需要申报一位家庭医生(Médecin Traitant),其一般情况下为全科医生,作为今后不管生什么病的一个主要联系人。家庭医生看病有问诊费,不过可以报销,且医生之后开的处方药也都是可以报销的。

首次选择和申报家庭医生需要下载后找到你所选择的医生共同签署一个申报表格Déclaration De Choix Du Médecin Traitant,寄给医保机构。但在选择你的家庭医生前,最好电话/邮件咨询一下,因为有一定数量的医生不再接受新患者来签订家庭医生。

在法国,就医需要预约。推荐一个全法通用的就医预约软件 Doctolib,既可以预约,还可以实现远程问诊。

\subsubsection{专科医生}
专科医生(Médecin Spécialiste)是专攻于某些领域的医生。当全科医生遇到无法解决的问题时会推荐一位专科医生。大部分专科医生需要全科医生的转诊,但是眼科医生和牙医一般不需要。

\subsubsection{校医}
对于在校生,可以选择校医。不同学校校医的报销情况有所不同,需要及时向学校咨询。

\subsubsection{药房}
如果得了一些小病,比如感冒、流鼻涕、喉咙痛、发热等等,大家觉得没有太大的必要去医院,就可以直接去家里附近的药店,药店的员工都是有专业资格证的,他们能够直接帮助解决这些小的病痛,而且能够及时推荐适合的药物。此外,药房还可以帮助注射疫苗(如HPV疫苗),但需要提前电话咨询。在法国,凡需凭医生处方出售的药物均由药房垄断经营。药房的绿色十字标志赫然醒目,每个街城市通常都有至少一个值班药房,以满足紧急需求。

\subsubsection{化验中心}
当医生开具处方需要你进行化验血检的话,可以前往化验机构(Laboratoire)。可以在谷歌地图搜索离家较近的化验中心,一般不需要预约,带好处方和医保卡前往即可。但如果不放心也可以选择约定时间。

\subsubsection{医院}
在法国医院分为公立医院、普通私立诊所和特约私立诊所,区别主要是在于看病的效率和报销的程度,当然看病的效率三种医院自然是从低到高,而报销的程度则是相反了。

公立医院里有开专门的急诊,如果是在晚上需要看诊可以直接去公立医院问诊,不过在公立医院问诊的话有一笔费用是需要自己处理并无法被报销的。如果大家在看急诊的过程中呼叫了救护车,这一笔费用也是需要从自己腰包里掏出来的哦,而且一般都是上百欧元了,所以呼叫救护车需谨慎。

大家如果真的遇到了紧急的问题,可以直接拨打紧急医疗救护处15进行咨询,拨打15的好处在于,他们知道你家附近所有的医疗状况,可以帮你推荐最佳的医院、诊所或者医生,而且在自己没有车的情况下他们也可以为你派车来你家接送,并且效率很高。

医疗参考:
\begin{itemize}
    \item \href{https://www.ecentime.com/article/ecentime-ameli-gratuit}{ECENTIME: 法国医保福利,免费体检?免费看牙?还有宫颈癌筛查项目!}
    \item \href{https://www.ecentime.com/article/comment-utiliser-doctolib-pendant-le-confinement}{ECENTIME: 宅家就医神器Doctolib}
    \item \href{https://www.dealmoon.fr/guide/1366}{ECENTIME: 法国常见病和就医常用句}
    \item \href{https://www.ecentime.com/article/2018-pharmacie-list}{ECENTIME: 急救药品指南:在法活着基本靠自己撑}
\end{itemize}

\subsection{文体娱乐}
\begin{itemize}
    \item 电影院、歌剧院、艺术节、艺术学校(音乐类和舞蹈类的 Conservatoire)
    \item 图书馆、博物馆、各类沙龙
    \item 游泳馆、健身房、各项其他体育运动的俱乐部(攀岩、瑜伽、马术、帆板、滑雪、击剑、搏击等)
\end{itemize}

推荐网址:
\begin{itemize}
    \item \href{https://www.ecentime.com/article/streaming-website }{ECENTIME: 看剧姿势不够优雅?法国视频网站大全}
    \item \href{https://paris-cine.info/}{Paris Cine Info 巴黎地区电影资讯}
    \item \href{https://www.operadeparis.fr/en}{Opera de Paris 巴黎歌剧院}
\end{itemize}

\subsection{旅游}
对于刚刚来到法国的同学们,理论上来说仅持学生签证、还没办好OFII的时候是不能出法国旅游的。在欧盟境内出入国境一般不会有海关检查,但是欧盟境内边检存在抽查(从经验中看德语区更严格),一旦被查出后果将会影响后续欧盟签证的办理,因此不建议大家冒险出境。

强烈建议大家积极参加法国清华校友会不定期组织的\href{https://www.tsinghua-france.org/category/activities/outdoor/}{户外活动}。

\subsection{打工}
\subsubsection{申请临时工作许可证}
拿学生长期签证或学生居留证,可以每年最多打964小时的工。如果超出这个时限,就必须由雇主在线申请临时工作许可证了。

\subsubsection{劳动合同}
劳动合同必须标明:职业名称,每月税前工资或者每小时税前工资,雇佣期限,工作地点,时间及起始日。

\subsubsection{学生工种}
找学生工的方法很多,可以关注学校,CROUS,超市或其他公共场所的广告栏,或者从地区的免费广告报纸上搜索,或者找常见的求职⽹站,例如Etudiant等或相关的中介Agence d’intérim等咨询。

鉴于学业空闲时间有限,比较适合大学生的工种有:教外语,照看小孩,餐馆、超市收银、翻译、散发广告、连锁快餐店侍应生,等等。博士生可以申请助教,在学校里带习题课(TD : Travaux dirigés)或者实验课(TP: Travaux pratiques)。

需要提醒⼤家的是,求职过程中也可能会遇到虚假诈骗信息,⼀定要仔细甄别。

\subsection{日常礼仪}
\subsubsection{积极融入当地文化}
踊跃参与课外活动,尝试结交新朋友,积极参与讨论和研究,以开放的心态结实志同道合的朋友。当地人喜爱出门社交,比如在酒吧聊天、去餐馆吃饭、一同参观展览或在家里待客。法国较中国更喜爱轻礼,登门与会或生日婚礼可携带例如花束、红酒或者巧克力等伴手礼。

\subsubsection{尝试学习法语}
法语是法国的官方语言,许多年轻人也有良好的英文水平。然而法语的支配地位决定了学习法语是能够增加你在法国生活厚度、深度和舒适度的最好捷径。不论是学习工作,还是看病、交友,基本的法语沟通都会为你带来便利,使你更容易感受到法国生活的魅力。学习方法:首选学校的法语课,其次每个市政府都有语言中心提供帮助,还有很多社会组织、商业学校可以提供课程。

\subsubsection{尊重当地政治、文化与宗教,求同存异}
法国的个人社会观点比较强烈,善辩论,与人辩论需秉承客观,避免过多的感情介入。

\subsubsection{Dos}
见面要说你好Bonjour, 待人接物要说谢谢Merci, 与他人交流要习惯眼神注视。

\subsubsection{Don'ts}
走路不要拉横排,排队不要与他人距离过近(尤其切记任何身体触碰),遇到可疑人员不要过度明显地绕道或者露怯以免激起坏人临时起意的作恶,不可将任何包裹(行李、书包等)单独留在公共空间(去洗手间、付费等)。

文化差异参考:\href{https://www.ecentime.com/article/difference-culturelle}{ECENTIME: 法中文化差异大盘点,来法留学的你一定要知道这些!}
