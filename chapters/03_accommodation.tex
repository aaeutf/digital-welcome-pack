% 抵达后住宿
\section{抵达后住宿}

注意,在法国,住房合同是人在法国生活最为重要的证明之一,很多手续都要求提供固定住址,很多文件也都会以信件的形式邮寄给你。因此,抵达后建议同学们尽快落实固定住址以便顺利办理银行、通讯、签证居留卡等事项。

如果有可能的话,请优先考虑学校安排的宿舍或公寓,这样可以避免自己找房子的麻烦及风险。

安顿后建议尽快更新收件箱姓名以便收取信件。传统邮政在法国仍然是重要的联系方式,很多重要信件都需要通过邮寄送达,所以一定要保证邮箱可以政策收件。

在入住后要仔细查看房间内的家具、装修是否有破损、孔洞等问题。如果有的话建议拍照,并立即和房东或管理员汇报,以免在退房时出现纠纷。

另外,要重视住宿的防火、防盗。检查好房间内是否安装了烟雾警报器,门锁是否结实可靠。重要的物品要想办法妥善保存。如果有疑虑可以在群里问校友的意见。

注意:法国的厨房一般通风性不是很好,烧烤、爆炒可能对家具及墙面造成无法恢复的损害,以至导致扣除押金等后果。建议避免大火炒菜。

\subsection{有接待方的同学}
与学校、奖学金机构、学术机构联系,获得居住地址。

\subsection{自行来法的同学}
\begin{itemize}
    \item 临时住宿(第一天):建议在抵法前找好住宿,否则,建议找安静街区的正规酒店作为第一落脚点。关于安静街区,详见7安全须知。
    \item 学生宿舍申请/押金/房租:在法国,有国家设立的学生住房 CROUS,和一些商业经营的学生公寓。通常在学校网站上会有关于该学校周边的住房资源信息,建议大家多多关注,及时申请。
    \item 私人房东/担保人/水电暖:关于自行寻找住宿,建议首选通过正规渠道(专门网站、正规的校园资源等)筛选,同时需要实地参观以检验社区邻里环境、房屋家具状态和交通便利程度等。参观住宿时要注意人身财产安全,不要轻易泄露个人信息。注意,要求先交定金再看房的通常都是诈骗。
    
    在法租房往往要求提供担保条件,包含1-2 个月押金和担保人银行信息担保,其中后者需涵盖3 倍房租的收入能力,往往可由父母、银行(付费)或专门的担保网站(付费)来出具。
    
    通常,租房的同时需要租客自行购买住房保险(参见5.5.3 住房保险Habitation),和开通水、电、暖气等的账户。有时水电暖费用会包含在租金里,需仔细阅读租房合约。
\end{itemize}

住宿参考:
\begin{itemize}
    \item \href{https://www.ecentime.com/article/location-logement}{ECENTIME: 法国租房,让你租到一个好房子必备知识帖}
    \item \href{https://www.ecentime.com/article/location-en-france}{ECENTIME: 法国租房最新干货帖,这些细节需要你注意!}
    \item \href{https://www.ecentime.com/article/connaissances-sur-etat-des-lieux}{ECENTIME: État de lieux,法国租房你必须知道的那些事!}
    \item \href{https://www.ecentime.com/article/location-en-france-tuto}{ECENTIME: 法国租房防被坑指南}
\end{itemize}
