\subsection{预防诈骗与隐私安全}

\subsubsection{电信诈骗}
但凡是以电话方式通知当事人涉案、要求转账汇款、回拨任何电话号码、通知护照过期或来使馆领取包裹或公文,都疑似诈骗。

不要在电话里向陌生人透露自己个人信息。

如遇陌生人来电要求“不要与家人和朋友联系,以保证他们的安全”,或提出其他不合常理的要求,应务必保持警惕,切勿上当受骗,要尽快通过其他渠道核实情况后再做处理。

如无法辨别是否为诈骗电话,建议挂断电话后拨打中国驻法国使领馆领事保护与协助电话,或拨打外交部全球领保与服务应急呼叫中心12308热线请求将相关信息转大使馆进一步核实。

如不幸上当受骗,应及时向法国警方报案,并同时向国内公安机关报警。

受害人无法直接向国内公安机关报案的,可通过国内近亲属及时报案,并向国内报案地反电信网络诈骗中心请求帮助(拨打110即可)。

在法国还会经常有推销性质的电话,一般来电显示是08开头,讲话者或多或少有一定口音,可以直接挂断。

\subsubsection{其他诈骗}
求职、换汇、盗号、回国带货等等。

\subsubsection{个人隐私安全}
网站众多,填写时应谨防个人信息泄露。时常更换密码。区分商业广告与工作生活邮箱。

防诈骗参考:
\begin{itemize}
    \item \href{http://www.amb-chine.fr/chn/sgxw/t1823322.htm}{AMB: 中国驻法大使馆在此提醒在法中国公民谨防电信诈骗}
    \item \href{https://www.ecentime.com/article/bien-vivre-en-france}{ECENTIME: 套路太多,防不胜防!法国留学生们请收下这篇防骗指南!}
\end{itemize}
