\documentclass[UTF8]{ctexart}

\usepackage{amsmath}
\usepackage{cases}
\usepackage{cite}
\usepackage{graphicx}
\usepackage[margin=1in]{geometry}
\geometry{a4paper}
\usepackage{fancyhdr}
\pagestyle{fancy}
\usepackage{hyperref}
\fancyhf{}

\usepackage[misc]{ifsym}

\newgeometry{margin=1in}
\newcommand\titleofdoc{\bfseries 法国清华校友会 数字迎新包}

\begin{document}

\begin{titlepage}
   \begin{center}
   
        \vspace{4 cm} 
        
        {\includegraphics[width=0.2\textwidth]{Logo.jpg}}\\

        \vspace{0.5 cm} 
        
        \noindent\rule[0.25\baselineskip]{\textwidth}{1pt}
        
        \vspace{0.5 cm} 
        
        \Huge{\titleofdoc} 

        \vspace{0.5cm}
        
        \LARGE{留学法国自助指南 第5版}
        
        \vspace{1 cm} 
        
        \noindent\rule[0.25\baselineskip]{\textwidth}{1pt}\\
        
        \vspace{1 cm}

        \large{\textbf{编辑:} 刁彩潇,陈远,康嘉,张凌睿,蔡长浩}\\

        \vspace{0.3 cm}

        \large{\textbf{更新日期:} 2024年7月19日}\\

        \vspace{1 cm}

        \large{扫码关注校友会的微信公众号,不要错过每一次活动!}\\

        \vspace{0.5 cm} 
        
        {\includegraphics[width=0.2\textwidth]{QRcode.png}}\\

        \vspace{1 cm} 

        \large{\textbf{写在前面:}建议刚到法国的校友\href{https://www.helloasso.com/associations/association-des-anciens-eleves-de-l-universite-tsi/adhesions/adhesion-2024-2025}{注册法国清华校友会2024-2025年度会员}(学生免费)。会员可以免费或低价参加校友会组织的各类活动及会议,并收到协会内部的信息(理事会会议记录、校友大会报告等)。}
      
    \end{center}
\setcounter{page}{0}
\end{titlepage}

\newpage
\tableofcontents

\fancyhead[C]{法国清华大学校友会\href{https://www.tsinghua-france.org/}{(AAEUTF)} \href{mailto:contact@tsinghua-france.org}{\textrm{\Letter} Email }}
\fancyfoot[C]{\thepage}

\newpage

\section{出国前行李准备}
\subsection{证件类}
护照(含签证),出生证明和双认证原件/翻译件/复印件,国内驾照原件/公证件/复印件(如有),学校录取信(包括地址与邮箱信息),毕业证和学位证和成绩单复印件及翻译件,证件照若干/电子版,在法住宿信息(地址,电话等),个人简历英文或法文版。

同时建议大家把这些文件扫描件保留电子档,和至少2份备用复印件分别放在不同行李。每个行李最好放一份机票和护照复印件,以备航空公司有行李延误或丢失的情况,方便找寻。保留好登机牌和行李牌。

\subsection{衣物类}

法国夏季一般比较凉爽,昼夜温差大。有时会出现持续几天的热浪,最高气温可以达到35-40度。冬季漫长但是不会很冷,很少会低于零下10度,但是阴雨天气居多。8月底、9月初到法国可适量准备夏装、秋装与冬装,至少准备一件保暖衣物以备应对短时降温。但是没有必要准备太厚重的衣服。

建议大家至少有一套正式着装以应对初到法国临时的正式场合:开学典礼、正式聚会等。不必太多,法国置装方便。

\subsection{日用品}

如非特殊需要,日常的洗漱用品在法国超市都很容易买到。所以,洗漱用品带旅行装足够抵达后前几日使用即可。女生可以注意,法国不常见双眼皮贴、假睫毛、假睫毛胶水、有色隐形眼镜和美瞳。

眼镜、有度数的隐形眼镜、雨伞、针线盒、指甲剪刀、剃须刀,有技术的可以带理发工具(法国理发价格相对较高)。

\subsection{电子产品}

转换插头(重要,注意欧盟国家种类繁多)、耳机、充电宝、手机电脑等设备及充电器、手机 sim卡换卡的小针等。电子配件强烈建议从国内携带(匹配度高价格更便宜)。如需换新,可在法国购买但价格略高(注意法国电脑默认键盘为法式键盘AZERTY,而非美式键盘QWERTY)。

注意:充电宝在是不允许放在飞机的托运行李里的。而且对充电量有最大要求,请留意你乘坐班机的具体规则。

\subsection{药品类}
法国医疗体系完善,药品齐全。建议大家自备些药品应对急性病症,如感冒药、跌打损伤类药物、止泻药、止痛药、烫伤膏等,记得保留说明书。

如因自身情况需要携带大量的处方药,找医生开好中英文证明,以备海关检查和不时之需。

个别抗生素在法国可能禁售或同类替代品剂量有差,建议大家遵从医嘱,提前了解欧洲替代品、或自行准备若干疗程,并携带医嘱。

注意,中药在欧洲不方便购买,且原材料不便过境海关,建议有需求的同学制成中成药后携带。

关于常见疾病疫苗,在法国会建议人在 25 岁时补充注射白喉、破伤风、脊髓灰质炎、百日咳等疫苗,如有常用疫苗手册,建议携带便于查阅接种历史。

\subsection{现金}
建议携带有VISA/Mastercard标示的银行卡,同时自备适量现金(落地交通/住宿也许需要)。请注意,非常不建议同学们随身携带大量现金,一方面跨境现金携带有数额上限(国境:价值20000人民币/5000美元,法境:10000欧元),另一方面为了避免同学们在境外遇到危险从而产生财产甚至人身伤害。

\subsection{关于床具}
如自带床品,需注意欧洲的寝具(床罩、枕套和被套等)的尺寸与大陆略有差异。在法国很容易购买日常床上用品。比如巴黎市区内就有\href{https://goo.gl/maps/zMbCesV2jhi92mcK9}{宜家IKEA},购买非常方便。

\subsection{关于书籍}

在法国,中文书籍资源有限,有纸质阅读或特殊工具书/专业书需求的同学建议自行携带。即使在巴黎,也只有少数中文书店(如凤凰书店Libraire Le Phénix、五楼书店wulolife、八梨空间8lithèque)可供消遣。

\subsection{关于厨具}
建议携带至少一双筷子。除非有特殊需求,在法国很容易购买日常厨房用具。中餐需要的电饭锅、炒锅如行李有余可携带小号,否则可在中国超市、线上超市购买,选择有限、价格略高,但可以满足基本需求。

\subsection{其它}
\begin{itemize}
    \item \href{https://www.ecentime.com/article/nvshengqingdan}{ECENTIME: 赴法留学行李女生篇}
    \item \href{https://www.ecentime.com/article/nanshengqingdan}{ECENTIME: 法国留学行李清单男生篇}
    \item \href{https://www.ecentime.com/article/qingdan}{ECENTIME: 留法行李清单分享!}
    \item \href{https://www.ecentime.com/article/daiqian}{ECENTIME: 第一次赴法到底怎么带钱啊!?没有信用卡咋整啊}
\end{itemize}

\newpage
\section{法国及校友会简介}
\subsection{历史及政治}

法兰西共和国起始于1789年法国大革命,法国的国庆日7月14日即攻占巴士底狱的日子。从此开始,法国经历了持续百年波澜壮阔的革命。

法兰西第一共和国的历史血雨腥风,大家耳熟能详的吉伦特派、雅各宾派、罗伯斯庇尔、热月政变等历史人物和事件,就是发生在这个时期。

法兰西第一共和国遭到了拿破仑的复辟,1804年拿破仑在创建了法兰西第一帝国。我们在卢浮宫里可以看到著名的《拿破仑一世加冕大典》油画,记录的就是巴黎圣母院里拿破仑加冕的场景。

此后法兰西经历了一系列的政变:1815年拿破仑在滑铁卢战败后波旁王朝复辟,波旁王朝又在1830年被七月革命推翻,建立了法兰西第二共和国。1851年拿破仑的侄子又重演叔叔的历史,推翻了第二共和国成立了法兰西第二帝国。第二帝国在1870年被推翻又成立了第三共和国,一直持续到二战的时候法国被德国占领。

二战中德国占领了巴黎及法国北部,并在南部成立了以维希(Vichy)为首都的伪政权(即维希法国)。戴高乐将军在沦陷的法国组织了抵抗运动(La Résistance),并在二战胜利后法国成立了第四共和国。今天的法兰西第五共和国,则是1958年经历了阿尔及利亚独立战争中的政变后由戴高乐建立的。

法兰西第五共和国是现代民主政体,其核心设计和原则来源于法国18世纪哲学家孟德斯鸠提出的三权分立,即立法、行政和司法三种国家权利互相制衡的思想。法国的立法机关由国民议会(Assemblée Nationale)及参议院(Sénat)两院组成,政府由总统(Présidant)和总理(Premier Ministre)组阁并领导,而司法系统主要由各种法院(Tribunal)组成。法国每一位公民都拥有直接选举国家总统、国民议会议员、市长、区长等权利。

法兰西共和国的价值观暨座右铭是“自由,平等,博爱(Liberté, Égalité et Fraternité)”,这三个词会出现在法国的很多建筑和标记中。法国宪法保证公民的人权和自由:包括言论自由、信仰自由、著作和出版自由、集会及结社自由等。可以毫不夸张的说,说法国的宪法经历了两个多世纪的发展和改良,已经成为了西方民主政治的典范。而这部宪法的核心价值和原则,对自由、民主和人权的追求,也成为了法兰西文化的一个重要组成部分。

今天的法国总统马克龙(Emmanuel Macron)是“一起为了共和国”(Ensemble pour la République)党(中间偏右派)的代表。值得注意的是,“一起为了共和国”在2024年6月初的欧洲议会选举中落败于极右翼政党“国民联盟”(Rassemblement national),随后马克龙宣布解散国民议会,并于2024年7月7日完成了新一轮议员选举。目前左派新人民阵线(Nouveau Front Populaire)获得国民议会中的最多议席,但无任何政党或联盟取得绝对多数。由于法国的宪法规定政府的组成要与国民议会的组成一致,所以在2025年下半年之前法国政府的组建和运作会变得更加复杂。

下一次总统选举会是在2027年。

\subsection{风俗及传统}
法国人喜爱出门社交,比如在酒吧聊天、去餐馆吃饭、一同参观展览或在家里待客。法国较中国更喜爱轻礼,登门与会或生日婚礼可携带例如花束、红酒或者巧克力等伴手礼。

关于宗教:目前法国社会盛行的主流宗教有基督教(天主教、新教和东正教)、伊斯兰教和犹太教。自 1905 年起,法国实行世俗化政策,严格执行政教分离政策,即对所有宗教保持中立,不存在国家官方宗教。此项政策确立了法国宗教多元化的架构,在近些年受到一些挑战和质疑。

法国现在还保留夏令时和冬令时,夏令时期间和中国时差6小时,冬令时期间和中国时差7小时。每年3月最后一个周日从冬令时改成夏令时,调前1小时。每年10最后一个周日从夏令时改成冬令时,调后1小时。

法国传统节日(法定假日):

• 元旦,Nouvel An,1月1日

• 复活节,Pâque,一般在3月底到4月

• 劳动节,Fête du Travail,5月1日

• 二战停战日,Armistice de la Seconde Guerre Mondiale,5月8日

• 耶稣升天节,Ascension,复活节过40天,一般在5月中旬

• 圣灵降临节,Pentecôte,复活节后第七个周日,一般在五月下旬到6月中旬。

• 国庆节,Fête Nationale,7月14日

• 圣母升天节,Assomption,8月15日

• 诸圣瞻礼节,Toussaint,11月1日

• 一战停战日,Armistice de la Première Guerre Mondiale,11月11日

• 圣诞节,Noël,12月15日

法国城市介绍:\href{https://www.ecentime.com/article/-france-geographie}{ECENTIME: 法国大区城市介绍|最值得去的城市有哪些,快来看看你要去哪个迷人的地方?} 

\subsection{清华大学法国校友会(AAEUTF)}

清华大学法国校友会 AAEUTF(Association des Anciens Elèves de l’Université Tsinghua en France),是由旅法工作、交流和留学的清华大学校友自愿组成的联合性、非营利性的群众组织,2009年9月18日注册,2009年10月7日正式成立于法国巴黎。协会主要宗旨是服务校友,服务母校,促进中法科技文化交流。

我们的宗旨是:服务广大旅居法国和留学法国的校友,促进校友间的情谊,加强校友与母校之间的联系和团结,继承发扬“自强不息,厚德载物”的校训精神和“严谨、勤奋、求实、创新”的优良传统,为母校的发展和成为世界一流水平的大学做贡献。经过11年多的锻炼成长,清华大学法国校友会已经变成一个成熟的、有经验、有影响力的组织。随着更多的校友来法深造发展,法国清华校友会的队伍还在不断壮大,影响也在继续提升,逐渐成为了中国留法学人最喜爱的校友会之一,清华人的精神在法兰西土地之上得到了发扬光大。

清华大学法国校友会理事会是校友会的决策机构,现任会长是康嘉,秘书长为刘大伟。此外,校友会在2022年开始还成立了顾问团,对理事会进行建议及沟通。

校友会相关信息:
\begin{itemize}
    \item 校友会官网:\href{https://www.tsinghua-france.org}{https://www.tsinghua-france.org}
    \item 校友会会员注册(学生免费):\href{https://www.tsinghua-france.org/about/registration/}{https://www.tsinghua-france.org/about/registration/}
    \item 理事会电子邮箱:secretariat@tsinghua-france.org 
    \item 校友会微信公众号:AAEUTF(二维码见封面页)
    \item 讨论群:我们在微信和Telegram上都有群组,你可以通过联系理事会加入。
\end{itemize}


\subsection{中国驻法国大使馆与学联}

中国驻法国大使馆于1964 年1 月 7 日中法建交后成立与法国巴黎,现任大使为卢沙野先生,使馆现地址为20, Rue Monsieur, 75007 Paris,使馆官网为\href{http://www.amb-chine.fr/chn/}{http://www.amb-chine.fr/chn/}。另在法国本土与海外省设5领馆:斯特拉斯堡、马赛、里昂、圣丹尼、帕皮提。留学在外,建议同学们时刻关注使馆动态发布。

另外,全法中国学者学生联合会(UCECF简称“全法学联”)是旅法中国留学人员自我管理,自我服务的群众性组织,是中国驻法使馆教育处联系广大留学人员的重要纽带。法学联在全法国各留学人员密集的城市均设有分学联, 覆盖了整个法国版图 (包括科西嘉岛)。全法学联会更新一版留学生新生手册,可以从微信公众号下载。全法学联官网:\href{www.ucecf.fr}{www.ucecf.fr},微信公众号:全法学联。

\newpage
\section{抵达后住宿}

注意,在法国,住房合同是人在法国生活最为重要的证明之一,很多手续都要求提供固定住址,很多文件也都会以信件的形式邮寄给你。因此,抵达后建议同学们尽快落实固定住址以便顺利办理银行、通讯、签证居留卡等事项。

如果有可能的话,请优先考虑学校安排的宿舍或公寓,这样可以避免自己找房子的麻烦及风险。

安顿后建议尽快更新收件箱姓名以便收取信件。传统邮政在法国仍然是重要的联系方式,很多重要信件都需要通过邮寄送达,所以一定要保证邮箱可以政策收件。

在入住后要仔细查看房间内的家具、装修是否有破损、孔洞等问题。如果有的话建议拍照,并立即和房东或管理员汇报,以免在退房时出现纠纷。

另外,要重视住宿的防火、防盗。检查好房间内是否安装了烟雾警报器,门锁是否结实可靠。重要的物品要想办法妥善保存。如果有疑虑可以在群里问校友的意见。

注意:法国的厨房一般通风性不是很好,烧烤、爆炒可能对家具及墙面造成无法恢复的损害,以至导致扣除押金等后果。建议避免大火炒菜。

\subsection{有接待方的同学}
与学校、奖学金机构、学术机构联系,获得居住地址。

\subsection{自行来法的同学}
\begin{itemize}
    \item 临时住宿(第一天):建议在抵法前找好住宿,否则,建议找安静街区的正规酒店作为第一落脚点。关于安静街区,详见7安全须知。
    \item 学生宿舍申请/押金/房租:在法国,有国家设立的学生住房 CROUS,和一些商业经营的学生公寓。通常在学校网站上会有关于该学校周边的住房资源信息,建议大家多多关注,及时申请。
    \item 私人房东/担保人/水电暖:关于自行寻找住宿,建议首选通过正规渠道(专门网站、正规的校园资源等)筛选,同时需要实地参观以检验社区邻里环境、房屋家具状态和交通便利程度等。参观住宿时要注意人身财产安全,不要轻易泄露个人信息。注意,要求先交定金再看房的通常都是诈骗。
    
    在法租房往往要求提供担保条件,包含1-2 个月押金和担保人银行信息担保,其中后者需涵盖3 倍房租的收入能力,往往可由父母、银行(付费)或专门的担保网站(付费)来出具。
    
    通常,租房的同时需要租客自行购买住房保险(参见5.5.3 住房保险Habitation),和开通水、电、暖气等的账户。有时水电暖费用会包含在租金里,需仔细阅读租房合约。
\end{itemize}

住宿参考:
\begin{itemize}
    \item \href{https://www.ecentime.com/article/location-logement}{ECENTIME: 法国租房,让你租到一个好房子必备知识帖}
    \item \href{https://www.ecentime.com/article/location-en-france}{ECENTIME: 法国租房最新干货帖,这些细节需要你注意!}
    \item \href{https://www.ecentime.com/article/connaissances-sur-etat-des-lieux}{ECENTIME: État de lieux,法国租房你必须知道的那些事!}
    \item \href{https://www.ecentime.com/article/location-en-france-tuto}{ECENTIME: 法国租房防被坑指南}
\end{itemize}

\newpage
\section{学校报到注册}
\subsection{关于留基委抵法报到流程}
已获得CSC奖学金且法国留学的小伙伴们,可参考以下攻略:\href{https://www.ecentime.com/article/CSC-China-Scholarship}{ECENTIME: 国家留学基金委奖学金CSC赴法报到攻略} 

\subsection{行政注册}
在收到录取通知书同时,你们有可能已经收到注册表,注册手续及注册所需材料清单。如果没有的话,可以联系所选的教学机构的注册处或教务处。

注册时间一般在9月,按学校指定日期前把填好的注册表、学生医疗保险购买凭证等相关材料和学费交回教务处,就可以拿到一个大学注册证明和学生证。

注册证明对于随后办理学生医保注册很重要,学生证对于出入学校、文娱场所票价减免很重要。

疫情期间,也许会有不同程度的变化,请大家关注学校动态,以各自院校规定为准。

\subsection{学业注册}
本科/硕士/博士或者工程师的学业选课,也是分必修和选修,博士选课需要到博士生院去咨询课目。请大家以各自院校规定为准。

\subsection{法语学习}

法语是法国的官方语言,许多年轻人也有良好的英文水平。

然而法语的支配地位决定了学习法语是能够增加你在法国生活厚度、深度和舒适度的最好捷径。不论是学习工作,还是看病、交友,基本的法语沟通都会为你带来便利,使你更容易感受到法国生活的魅力。

学习方法:首选学校的法语课,其次每个城市都有语言中心提供系统教学课程(学费相对较贵),还有很多社会组织、商业学校可以提供课程。

有些学校或专业会要求留学生达到相对应的法语学习文凭DELF作为毕业条件之一。

法语参考:\href{https://www.ecentime.com/article/francais-quotidien-}{ECENTIME: Bonjour! 首次赴法很焦虑?掌握这些法国生活基本用语吧 }

\newpage
\section{抵法初期行政手续}

在法国,办理行政手续是每一位留学生在法生活印象深刻的必修课,需要大家给予一定的\textbf{耐心、信心和同情心}。

刚刚抵法后的行政手续办理逻辑如下:
\begin{itemize}
    \item 办理一张预付费手机卡,拥有法国手机号
    \item 确定可收取信件的固定住址,签订住房合同
    \item 凭借住房证明方可办理法国银行卡、办理签证生效OFII
    \item 凭借办理银行卡时获得的账户信息(RIB/IBAN)方可办理住房补助(CAF)、通讯手机卡、网络盒子、水电账户、学生保险等(注意无需等银行卡到手)
    \item 凭借法国手机号方可顺利办理所有其他手续(收验证码等)
    \item 凭借签证生效OFII、银行卡等注册Ameli(医保)账号、申请Vitale卡(医保卡)
\end{itemize}

行政参考:\href{https://www.ecentime.com/article/demarches-etudiant-arrivant}{ECENTIME: 法国行政手续,接好这份抵法后各种手续的办理顺序整理!}
 
\subsection{签证换居留证OFII}
在法居留证明是办理一切行政手续的必须材料。对于第一年的学生来说是护照页上贴的OFII,第一年过后是居留卡Titre de Séjour。

相信大家来法国前一定已经持有申根-法国签证。如果你持有的是6-12个月的长期签证,需在抵达法国后三个月内到法国移民局办理“签证生效手续”,又称OFII。 这一行政步骤是强制性的,以保证你在法国的法律地位。未能完成此步骤将导致非正常移民身份,您可能将无法越过申根边境,并会影响你办理后续的其他行政手续或享受法国福利(如CAF申请长居等)。

自2019年起,本业务可在线办理。办理业务需要准备法国住房合同和印花税邮票,后者可在直接在线购买或在附近烟草店Tabac购买,或线上购买。

具体办理流程可自行前往\href{https://administration-etrangers-en-france.interieur.gouv.fr}{法国移民局网站}办理:,网页右上角有选项可切换语言。

OFII参考:\href{https://www.ecentime.com/article/tuto-ofii}{ECENTIME: 法国签证生效手续怎么申请?OFII可以在网上办理啦!}
 
\subsection{银行账户}

法国以信用卡为主,不通用借记卡。主流的信用卡厂家是 Carte Blue/VISA/MasterCard,主要银行有BNP Paribas, Société Générale, Caisse Epargne, LCL, Crédit Agricole 等。银行开户要求提供住房合同,所以务必先确认自己的住房合同无误后再去办理。除此以外,还需携带护照和学生注册证明。有些银行会在学校刚开学时提供便捷服务和福利,对于学生有免年费的优惠政策,具体可以在开户时详细咨询。

开户的过程中,一定要咨询所办理银行卡的消费额度(Plafond),以及每周的取现限额。此外,如果有办理住房保险的需求,也可以直接向银行的账户顾问(Conseiller)咨询相关服务。建议同时办理银行卡与支票本。办妥当天可获得账户信息(RIB/IBAN,重要!)用于办理后续手续。开户后将用过信件在登记的固定住址分别收取银行卡和密码,办理时长为自办理日起预计 2-3星期用于制卡、邮寄。

法国是世界上为数不多的保留支票(chèque)作为主要支付手段的国家。在法国,因为现金的管制很严格,所以大额的交易一般都是用支票支付的,比如学费、房租、押金,甚至不少公司实习期间的工资也会开支票。 关于开支票付费与存支票收款,我们在网上为大家找了几篇讲解详细的攻略供大家参考。

银行参考:
\begin{itemize}
    \item \href{https://www.ecentime.com/article/creation-compte-bancaire}{ECENTIME: 法国银行开户,最全攻略敬请查收}
\end{itemize}

\subsection{手机}

法国有四家主要通讯运营商:Bouygues, SFR, Orange, Free. 法国的电话卡资费套餐通常都包含接打、短信和欧盟通用的上网流量。关于各家的优惠套餐,建议大家实时关注。

虽然去银行开户后便可凭借账户信息RIB/IBAN和其他材料去运营公司办理固定手机卡,但是由于开户和收取电话卡前后至少需等待2-5工作日不等,强烈推荐同学们落地后手中至少有一张预付费临时电话卡(Carte Prépayée)或者其他可以在法使用一周左右的手机卡,否则落地后无法立刻拥有固定手机卡与家人联系,银行开户时也需要预留手机号,且银行开户后去办手机卡需要等待邮寄。

出国旅游时,每到一个国家,都会收到运营商发出的关于当前国家的资费短信(收短信都是免费哒),告知打电话,发短信,用流量等所需费用。欧盟境内一般有一部分共享流量,建议大家提前了解自己的话费套餐是否包含出国项目,尤其是流量。如果不包含,某些运营商提供临时漫游流量套餐等,最好提前购买。

通讯参考:
\begin{itemize}
    \item \href{https://www.ecentime.com/article/forfait-mobile-france}{ECENTIME: 法国电信服务大盘点}
    \item \href{https://www.ecentime.com/article/carte-sim}{ECENTIME: 办手机卡的那些事儿}
\end{itemize}

\subsection{网络}

在法国,家庭网络的使用一般是通过一种三网合一的网络盒子(Box)完成的,可以同时提供电信网络(固定电话)、计算机网络和有线电视网络。网络技术一般分为传统的同轴电缆(ADSL),或者更高速的光纤(Fibre)。目前,法国常见的网络运营商有Orange/SFR/Bouygues/Free,分别拥有自己的网络套餐。如果住宿不包含网络的话,需要同学们自行选择一家运营商开始签订网络套餐。

办理时,首先需要你已经办理好银行开户,持有账户信息(RIB或者IBAN)。其次,你需要知道所在房间是否曾经接入过某个运营商来决定是需要架设线路还是激活休眠线路。这个信息就需要大家从房东那里询问,或者在以上4家主流运营商网站上通过各自数据库来查找。

如果找不到所在住所的网络线路信息,那么有大几率这个住所还没有架设网络线路,所以需要选择运营商和套餐,预约修理工上门架设,这个过程会收费且需要预约。

如果查找到或者已知从前房子已经办理过网络,那么仅需网上办理激活,同时需要提供上一任使用该线路的住客姓氏。一般来说,可通过房东获得或者以上几家运营商为争夺客户会主动帮助查找上任住客信息。

网络参考:
\begin{itemize}
    \item \href{https://www.ecentime.com/article/network-box-in-france}{ECENTIME: 网盒,法国BOX办理没头绪?从办理到使用全解答!}
\end{itemize}

\subsection{保险}
\subsubsection{社会医疗保险}

法国的医保体制由两部分组成:基本医疗保险以及补充险。基本医疗保险由法国医保部门Assurance maladies负责任何在法国拥有合法拘留权的人都包括在内。费用视你的收入而定。基本险一般报销各类诊疗费用的60\%-70\%。余下部分的报销就由补充险赔付。

法国学生医疗保险制度将进行改革,新推出的CVEC(La Contribution Vie Etudiante et de Camps)取代了原来的学生社会保险(La Sécurité Sociale Etudiante), 以90欧/年获得医保卡(Carte Vitale), 享受医疗保障包括门诊、就医、买药等最高70\%的报销额度。购买可通过\href{https://www.messervices.etudiant.gouv.fr}{网站}进行。付款成功后下载保留注册证明,在去学校注册时需要提供。购买成功后需要要在社保网站注册生效。

法国行政效率较为缓慢,可以通过电话沟通加快获得Carte Vitale和激活ameli账户的速度。\href{https://www.ameli.fr/paris/assure/english-pages}{电话参考}:09 74 75 36 46。

所需材料:学校注册证明、护照、学生签证、出生公证、居留证/OFII、银行账户信息RIB。

社保参考:\href{https://www.ecentime.com/article/assurance-maladie}{ECENTIME: 法国医保、学生医保不会办理?保姆级申请攻略你值得拥有!}
 
\subsubsection{辅助保险Mutuelle – LMDE/MGEN/Matmut}

法国医保的第二部分是补充险(Mutuelle),不是必须购买的,这部分由各大保险公司负责,病人可以自由选择保险公司。费用和报销比例视你自由选择的套餐而定。对于留学生,学校一般会为大家推荐购买学生医疗保险。

在法国看牙医或配眼镜,如果没有投保补充医疗保险的话,普通社会医疗保险机构的报销水平则会较差。

\subsubsection{CSS补充险}
对于低收入人群(如学生)可以申请CSS补充险,可参考\href{https://www.xiaohongshu.com/explore/6244b05d00000000210386be?note_flow_source=wechat}{链接}。

\subsubsection{住房保险Habitation} 
房屋保险是租房必须的手续之一。和房东或中介签署房屋合同前,明确是否需要你另外购买一份房屋保险。

如需购买,可在办理法国银行卡的同时在开卡银行咨询购买房屋保险的业务,或去专业的商业保险公司购买。

常见的商业保险公司有MAAF/MAE/Groupama等。

\subsection{住房补助CAF}
	
法国的房补政策是一种法国特有的社会福利,面向所有居住在法国的特定人群(包含学生)。强烈建议同学们抵达法国并获得房屋合同后尽快办理 CAF以便尽早享受房补。

按照规定,CAF自开户提交申请之日算起,且抵达法国的第一个月和彻底离开法国前的最后一个月是没有房补的。比如:

1). 8月份入境法国,房屋合同从8月或更早开始,如果在8月底前建立CAF账户并申请房补,第一笔房补就是从9月开始算的。 

2). 8月份入境法国,房屋合同从9月开始,即使在9月份建立CAF账户并申请房补,也只能从10月开始拿房补,因为9月是住房合同生效的第一个月。 

3). 8月份入境法国,房屋合同从8月或更早开始,但一直拖到10月份才建立CAF账户并申请房补,第一笔房补也只能从10月开始算。

虽然CAF的材料中要求提交银行卡RIB以及OFII证明,但这些材料是\textbf{可以之后补充}的,补材料的递交时间不影响第一笔房补的发放时间。入境时间、住房合同开始时间、建立CAF账户并申请房补的时间这三个因素决定了第一笔房补的起算月份。

CAF参考:\href{https://www.ecentime.com/article/demande-caf}{ECENTIME: CAF申请难到哭,详细到感动的申请指南来啦!}
 
\subsection{水电账户}

初到法国的小伙伴们在搬进住所之后可能会面临需要新办理用电账户的问题。法国比较常见的供电公司包括:EDF、Direct Energie、ENGIE等。

电账户可选择自行开户或从房东/上任房客那里过户,办理需要提供以下材料:有效的居留证件(带有学生签证的护照)、银行RIB、住房合同、上一任住户的信息和电表读数、电表设备号码。其中后面两项可咨询房东和供电公司。

电账户参考:
\begin{itemize}
    \item \href{https://www.ecentime.com/article/edf}{ECENTIME: 法国电力公司EDF办理保姆级教程来啦,手把手教你开户!}
    
    \item \href{https://www.ecentime.com/article/guide-electricite-comparaison-prix-bas}{ECENTIME: 开电指南,货比三家,办理迅速又省钱}
    
    \item \href{https://www.ecentime.com/article/edf-box-internet}{ECENTIME: 法国水电网开户办理流程!}
\end{itemize}

\subsection{中国大使馆教育处报到}

对于今后有回国就业规划的同学们,我们在海外的同学们需要注意一项关于学历学位认证的手续。首先,大家需在抵法后三个月内完成“留学人员登记”,随后在回国前要准备《留学回国人员证明》,最后,回国后要申请学位学历认证。由于使馆教育处无法联系到每一个来法人员,因此同学们需要自发完成登记备案工作,这样即使在今后出现人员失联、紧急案件等都能够凭备案信息联系家属,有备无患。具体登记通道为:\href{http://www.education-ambchine.org/}{http://www.education-ambchine.org/}。强烈建议大家在完成后牢记账号密码,因为到回国的时候可能由于时间久远加上手机号邮箱等的更新导致找回密码困难。

\subsection{报税}
在法国,每年四月是法国的报税月。虽然学生没有固定收入,仍然建议大家主动申报收入,因为报税年数越多,对于今后转换工作签证等越有帮助。另外报税也是证明自己是低收入或者无收入群体的必要步骤,有助于同学们申请CSS补充保险和\href{https://www.solidaritetransport.fr/}{Carte de Transport Solidaire}(一种福利性质的交通卡,每月18欧)。对于刚来法国的小伙伴,第一次报税需要在法国政府税务网站上开户,可在官网在线开户,或者去当地的税务局领取报税表格。

税务网址:\href{https://www.impots.gouv.fr/portail/}{https://www.impots.gouv.fr/portail/} 

在线开户需要填写的表格:
\href{https://www.impots.gouv.fr/portail/contacts?778}{https://www.impots.gouv.fr/portail/contacts?778}

报税参考:
\href{https://www.ecentime.com/article/ecentime-impots-france}{ECENTIME: 法国第一次报税}

\newpage
\section{生活}
生活参考:\href{https://www.ecentime.com/article/liste-app}{ECENTIME: 无缝衔接中法生活,你需要这些App!}


\subsection{超市}
	
法国超市品牌有Carrefour, Géant, Auchan, Franprix, Lidl, Casino, SuperU, Intermarché等,此外还有有机生活超市(法国的有机食品会标注“bio”)、市场。另外出于减少食物浪费和环境保护的考虑,一些商家(除超市外还有面包甜品店、餐馆等)会将当天无法售出但依旧新鲜的食物以较低价格上架到TooGoodToGo软件平台上,俗称“剩菜盲盒”。

实体中超品牌主要集中在大巴黎地区。巴黎有\href{https://www.tang-freres.fr/}{陈氏(Tang Frère)}、\href{https://paris-store.com/}{巴黎士多(Pairs Store)}、\href{https://www.instagram.com/chenmarketfr/?hl=en}{中国红(Chen Market)}、新今日、大中华等。韩超有\href{https://www.instagram.com/kmartfrance/?hl=en}{K-Mart}。

里尔地区:

\href{https://maps.app.goo.gl/arekUkxV9KnKrmaj8}{Asie Nord, 15-17 Rue Jules Guesde, 59000 Lille}

\href{https://maps.app.goo.gl/4teSNmTntTzK39y36}{Paris Store, 23 Rue du Collège, 59100 Roubaix}

线上中超软件包括\href{https://mywaysia.com/en}{方圆食里Waysia}(校友经营,可送至巴黎以外地区)、悟空送菜(巴黎地区)、打酱油(德国可外送法国)等,他们提供中国食材和24$\times$7外送服务。

超市参考:
\begin{itemize}
    \item \href{https://www.ecentime.com/article/supermarket}{ECENTIME: 超市知多少——法国生活必备的超市有哪些?}
    \item \href{https://www.ecentime.com/article/ecentime-supermarketfrance}{ECENTIME: 超市会员卡大测评}
\end{itemize}


\subsection{饮食}

\subsubsection{大学食堂}
大学食堂CROUS是法国政府补贴的学生食堂,价格低廉、餐食简单。全法国共有800多个大学食堂。法国的大学餐厅 (Restaurants universitaires) 供应的菜式大抵依前菜、主菜、点心等三四道菜配成,面包可随意取用,饮料须另外购买。

首先在你的Crous卡(类似中国大学的校园一卡通)或是学生证里充钱,校园内可能有可以冲卡的机器(Borne),直接用银行卡充值即可。或者使用Izly软件充值。食堂价格在各城市会有差异,举例来说,巴黎一家CROUS(3.30欧)的一餐包括6个点(Point),按照点数自助选取。一般一份主菜Plat在3-4个点;沙拉1个点;水果1个点;汤类1个点;甜点1-2个点。绝大多数Resto U只在午餐时间开放(11:30 - 14:30)。

\subsubsection{餐馆酒吧}
餐馆以法餐为主,主要分为星级餐馆(étoile),普通餐馆(Brasserie/Bouillon/Bistro等),简食餐厅(Crêperie /Bar à salade等)。另有各国料理(日韩、印度、泰国、意式、美式、北非及中东地区美食等等)。在法国有越来越多的中餐馆,从盒饭快餐厅(Traiteur Chinois)到与国内接轨的地方特色美食一应俱全。巴黎地区中餐餐馆选择众多,大部分可外送至巴黎。

里尔地区:\href{https://maps.app.goo.gl/ScHQq7jbV4Bug9JJ7}{Tsingtao烤鸭店 13 Rue Jules Guesde, 59000 Lille}

在法国,随处可见咖啡厅和酒馆,是绝大部分法国人十分青睐的与会友闲聊首选场所。同时也慢慢有越来越多的特色奶茶店与甜品店可供国人聚会消遣。

餐饮参考:
\begin{itemize}
    \item \href{https://www.ecentime.com/article/comment-profider-le-crous-en-france}{ECENTIME: 法国食堂Crous你还没好好利用起来么?}
    
    \item \href{https://www.ecentime.com/article/ecentime-bar-paris}{ECENTIME: 巴黎酒吧周末好去处}
\end{itemize}

\subsection{垃圾分类}

法国的垃圾需要严格分类才可以丢弃,随意丢弃垃圾可能会收到罚款。常见的垃圾分为:

\begin{itemize}
    \item 可回收垃圾(Déchets recyclables):包装盒、塑料袋、塑料瓶、金属等散装放入黄色的垃圾桶。玻璃瓶放入玻璃制品特定的垃圾桶。
    \item 普通垃圾:无法回收的家庭垃圾,放入密封垃圾袋丢到灰色或棕色垃圾桶。
    \item 有害垃圾:包括药品、电池、电器、灯管等有害垃圾需要丢弃到专门的回收地点,请查看居住城市的相关规定。
    \item 废旧家具:请查看居住城市的相关规定,一般情况下每个月或每个礼拜会有固定的时间统一回收。
    \item 装修垃圾:需要自行送到垃圾处理厂,请查看居住城市的相关规定。
\end{itemize}

\subsection{交通}

\subsubsection{自行车}
在法国,全新的自行车大概平时价格在200欧左右,可在自行车商店或法国运动品商店迪卡侬Decathlon购买。二手自行车可以通过法国二手商品网站\href{https://www.leboncoin.fr/}{Leboncoin}查找购买。

另外,想临时骑车代步的小伙伴可查询所在城市是否有公共自行车,如巴黎的\href{http://www.velib.paris/}{Vélib},里尔的\href{https://www.lillemetropole.fr/votre-quotidien/se-deplacer/se-deplacer-velo}{V’lille} 

\subsubsection{市内交通}
于巴黎戴高乐机场,抵达后有不少中文指示牌,且有信息服务台和工作人员提供指引。前往市区可乘坐快轨(约 14 欧)、巴士(约12欧)或出租车(50-60欧固定费用)。

关于巴黎市内交通:有公共交通例如地铁(Metro)/快轨(RER)/小火车(Train)/电车(Tram)等,有共享交通例如巴黎自行车、脚踏车等的租赁。由于城市对车辆限制逐渐升级,道路限行/限速使得市内公共交通更受欢迎。必要时,可预约出租车G7、网约车Uber、Bolt等。

\subsubsection{SNCF/青年卡}

法国国家铁路公司SNCF是小伙伴们在法国出远门最依赖的交通方式,有高速铁路(TGV)贯穿全国、方便快捷省际列车TER直达相邻城市、低价的非高峰高铁Ouigo专列。而SNCF的票价浮动大不稳定,各式优惠卡、会员卡则能帮大家省钱:有适合频繁乘坐高铁的月卡TGV Max(27岁及以下)、年卡Forfait Annuel TGV,还有适合年轻人的青年卡(年卡)Carte Avantage Jeune(26岁及以下)等等,详情请见\href{https://www.ecentime.com/article/avantage-carte-sncf}{Link}。

交通参考:
\begin{itemize}
    \item \href{https://www.ecentime.com/article/aeroport-CDG%202020}{ECENTIME: 法国戴高乐机场CDG完全指南}
    \item \href{https://www.ecentime.com/article/transport-paris}{ECENTIME: 巴黎交通防丢指南}
    \item \href{https://www.ecentime.com/article/reduction-transport-etudiant}{ECENTIME: 学生公交卡办理攻略:巴黎、斯堡\&里昂的学生们看过来啦!}
    \item \href{https://www.ecentime.com/article/ecentime-sncf-reduction}{ECENTIME: SNCF最新优惠卡政策对比}
\end{itemize}

\subsubsection{Carte de Transport Solidaire}
通过报税证明自己是低收入或者无收入群体,并申请成功申请CSS后,同学们可以申请\href{https://www.solidaritetransport.fr/}{Carte de Transport Solidaire}(一种福利性质的交通卡,每月18欧)。

\subsubsection{自驾}
持学⽣长居卡的⼩伙伴,可直接使用经认证翻译的中国驾照。但是由于中法间交规有别,建议有驾驶经验的⼩伙伴出发前⼀定要确保驾照可用并注意遵守法国交规,安全驾驶!

\subsection{购物}
	
巴黎地区的商场:老佛爷(Galerie Lafayette), 巴黎春天(Printemps)、BHV、乐蓬马歇(Le Bon Marche)、新开的Sanmaritaine, 巴黎中心 Chatelet 地铁站的Westfield,拉德芳斯商务区的 Quatres Temps,巴黎郊区的打折村 Vallée Village等。另有多条购物街区:6区的Rue de four, 4区的玛黑区,11区外玛黑区等。

里尔:巴黎春天(Printemps)、EuraLille

法国的打折季每年两季,其中夏季打折季每年六月底七月初开始,冬季打折季每年一月开始。

网购参考:
\begin{itemize}
    \item \href{https://www.ecentime.com/article/ecommerce-fr-top15}{ECENTIME: 逛啥淘宝!法国网购电商TOP15出炉!}
    \item \href{https://www.ecentime.com/article/online-shopping-delivery}{ECENTIME: 法国网购邮寄方式汇总}
    \item \href{https://www.ecentime.com/article/amazon-fr-skill-1}{ECENTIME: 亚马逊Prime法国攻略及隐藏福利}
\end{itemize}

出于环保和节约的考虑,法国的二手文化较为流行。常见的网站和平台有\href{https://www.leboncoin.fr/}{leboncoin}、\href{https://www.vinted.fr/}{Vinted}(二手衣物)、\href{https://www.momox-shop.fr/}{momox}(二手书籍)等。


\subsection{医疗}

在法国,只要大家办理好了Carte Vitale医保卡,看病不难也不贵。

\subsubsection{全科医生}	
全科医生(Médecin Généraliste)提供基本的疾病诊疗和全面检查。在法国,每个享受医保的人需要申报一位家庭医生(Médecin Traitant),其一般情况下为全科医生,作为今后不管生什么病的一个主要联系人。家庭医生看病有问诊费,不过可以报销,且医生之后开的处方药也都是可以报销的。

首次选择和申报家庭医生需要下载后找到你所选择的医生共同签署一个申报表格Déclaration De Choix Du Médecin Traitant,寄给医保机构。但在选择你的家庭医生前,最好电话/邮件咨询一下,因为有一定数量的医生不再接受新患者来签订家庭医生。

在法国,就医需要预约。推荐一个全法通用的就医预约软件 Doctolib,既可以预约,还可以实现远程问诊。

\subsubsection{专科医生}
专科医生(Médecin Spécialiste)是专攻于某些领域的医生。当全科医生遇到无法解决的问题时会推荐一位专科医生。大部分专科医生需要全科医生的转诊,但是眼科医生和牙医一般不需要。

\subsubsection{校医}
对于在校生,可以选择校医。不同学校校医的报销情况有所不同,需要及时向学校咨询。

\subsubsection{药房}
如果得了一些小病,比如感冒、流鼻涕、喉咙痛、发热等等,大家觉得没有太大的必要去医院,就可以直接去家里附近的药店,药店的员工都是有专业资格证的,他们能够直接帮助解决这些小的病痛,而且能够及时推荐适合的药物。此外,药房还可以帮助注射疫苗(如HPV疫苗),但需要提前电话咨询。在法国,凡需凭医生处方出售的药物均由药房垄断经营。药房的绿色十字标志赫然醒目,每个街城市通常都有至少一个值班药房,以满足紧急需求。

\subsubsection{化验中心}
当医生开具处方需要你进行化验血检的话,可以前往化验机构(Laboratoire)。可以在谷歌地图搜索离家较近的化验中心,一般不需要预约,带好处方和医保卡前往即可。但如果不放心也可以选择约定时间。

\subsubsection{医院}
在法国医院分为公立医院、普通私立诊所和特约私立诊所,区别主要是在于看病的效率和报销的程度,当然看病的效率三种医院自然是从低到高,而报销的程度则是相反了。

公立医院里有开专门的急诊,如果是在晚上需要看诊可以直接去公立医院问诊,不过在公立医院问诊的话有一笔费用是需要自己处理并无法被报销的。如果大家在看急诊的过程中呼叫了救护车,这一笔费用也是需要从自己腰包里掏出来的哦,而且一般都是上百欧元了,所以呼叫救护车需谨慎。

大家如果真的遇到了紧急的问题,可以直接拨打紧急医疗救护处15进行咨询,拨打15的好处在于,他们知道你家附近所有的医疗状况,可以帮你推荐最佳的医院、诊所或者医生,而且在自己没有车的情况下他们也可以为你派车来你家接送,并且效率很高。

医疗参考:
\begin{itemize}
    \item \href{https://www.ecentime.com/article/ecentime-ameli-gratuit}{ECENTIME: 法国医保福利,免费体检?免费看牙?还有宫颈癌筛查项目!}
    \item \href{https://www.ecentime.com/article/comment-utiliser-doctolib-pendant-le-confinement}{ECENTIME: 宅家就医神器Doctolib}
    \item \href{https://www.dealmoon.fr/guide/1366}{ECENTIME: 法国常见病和就医常用句}
    \item \href{https://www.ecentime.com/article/2018-pharmacie-list}{ECENTIME: 急救药品指南:在法活着基本靠自己撑}
\end{itemize}

\subsection{文体娱乐}
\begin{itemize}
    \item 电影院、歌剧院、艺术节、艺术学校(音乐类和舞蹈类的 Conservatoire)
    \item 图书馆、博物馆、各类沙龙
    \item 游泳馆、健身房、各项其他体育运动的俱乐部(攀岩、瑜伽、马术、帆板、滑雪、击剑、搏击等)
\end{itemize}

推荐网址:
\begin{itemize}
    \item \href{https://www.ecentime.com/article/streaming-website }{ECENTIME: 看剧姿势不够优雅?法国视频网站大全}
    \item \href{https://paris-cine.info/}{Paris Cine Info 巴黎地区电影资讯}
    \item \href{https://www.operadeparis.fr/en}{Opera de Paris 巴黎歌剧院}
\end{itemize}

\subsection{旅游}
对于刚刚来到法国的同学们,理论上来说仅持学生签证、还没办好OFII的时候是不能出法国旅游的。在欧盟境内出入国境一般不会有海关检查,但是欧盟境内边检存在抽查(从经验中看德语区更严格),一旦被查出后果将会影响后续欧盟签证的办理,因此不建议大家冒险出境。

强烈建议大家积极参加法国清华校友会不定期组织的\href{https://www.tsinghua-france.org/category/activities/outdoor/}{户外活动}。

\subsection{打工}
\subsubsection{申请临时工作许可证}
拿学生长期签证或学生居留证,可以每年最多打964小时的工。如果超出这个时限,就必须由雇主在线申请临时工作许可证了。

\subsubsection{劳动合同}
劳动合同必须标明:职业名称,每月税前工资或者每小时税前工资,雇佣期限,工作地点,时间及起始日。

\subsubsection{学生工种}
找学生工的方法很多,可以关注学校,CROUS,超市或其他公共场所的广告栏,或者从地区的免费广告报纸上搜索,或者找常见的求职⽹站,例如Etudiant等或相关的中介Agence d’intérim等咨询。

鉴于学业空闲时间有限,比较适合大学生的工种有:教外语,照看小孩,餐馆、超市收银、翻译、散发广告、连锁快餐店侍应生,等等。博士生可以申请助教,在学校里带习题课(TD : Travaux dirigés)或者实验课(TP: Travaux pratiques)。

需要提醒⼤家的是,求职过程中也可能会遇到虚假诈骗信息,⼀定要仔细甄别。

\subsection{日常礼仪}
\subsubsection{积极融入当地文化}
踊跃参与课外活动,尝试结交新朋友,积极参与讨论和研究,以开放的心态结实志同道合的朋友。当地人喜爱出门社交,比如在酒吧聊天、去餐馆吃饭、一同参观展览或在家里待客。法国较中国更喜爱轻礼,登门与会或生日婚礼可携带例如花束、红酒或者巧克力等伴手礼。

\subsubsection{尝试学习法语}
法语是法国的官方语言,许多年轻人也有良好的英文水平。然而法语的支配地位决定了学习法语是能够增加你在法国生活厚度、深度和舒适度的最好捷径。不论是学习工作,还是看病、交友,基本的法语沟通都会为你带来便利,使你更容易感受到法国生活的魅力。学习方法:首选学校的法语课,其次每个市政府都有语言中心提供帮助,还有很多社会组织、商业学校可以提供课程。

\subsubsection{尊重当地政治、文化与宗教,求同存异}
法国的个人社会观点比较强烈,善辩论,与人辩论需秉承客观,避免过多的感情介入。

\subsubsection{Dos}
见面要说你好Bonjour, 待人接物要说谢谢Merci, 与他人交流要习惯眼神注视。

\subsubsection{Don'ts}
走路不要拉横排,排队不要与他人距离过近(尤其切记任何身体触碰),遇到可疑人员不要过度明显地绕道或者露怯以免激起坏人临时起意的作恶,不可将任何包裹(行李、书包等)单独留在公共空间(去洗手间、付费等)。

文化差异参考:\href{https://www.ecentime.com/article/difference-culturelle}{ECENTIME: 法中文化差异大盘点,来法留学的你一定要知道这些!}

\newpage
\section{安全须知}
\subsection{自我保护}
出门不要带太多现金,一般二三十欧现金足够,建议用银行卡代替现金。

步行时要注意身边的环境,如是否有神色可疑的人观察你、是否有人跟踪你。不建议边走路边听歌、打电话或是玩手机,在这些情况下容易放松警惕。在对当地情况熟悉前,最好能结伴而行,尽量避免行经人少幽暗之处,不要抄捷径、走小路,并在外出前告知同学或者父母。

乘坐公共交通时,须时刻看管好行李和随身携带的包,同时尽量与旁边的人保持一定的距离。尽量不要让陌生人在你身后和你一起过检票闸(逃票),以防被顺手牵走钱包、手机等。

了解所在城市的安全区域与安全隐患,尽量避免独自前往安全堪忧区域。

牢记报警与使馆联络方式。

安全参考:\href{https://www.ecentime.com/article/GuideUrgenceFrance}{ECENTIME: 法国应急指南}
 

\subsection{预防诈骗与隐私安全}
\subsubsection{电信诈骗}
但凡是以电话方式通知当事人涉案、要求转账汇款、回拨任何电话号码、通知护照过期或来使馆领取包裹或公文,都疑似诈骗。

不要在电话里向陌生人透露自己个人信息。

如遇陌生人来电要求“不要与家人和朋友联系,以保证他们的安全”,或提出其他不合常理的要求,应务必保持警惕,切勿上当受骗,要尽快通过其他渠道核实情况后再做处理。

如无法辨别是否为诈骗电话,建议挂断电话后拨打中国驻法国使领馆领事保护与协助电话,或拨打外交部全球领保与服务应急呼叫中心12308热线请求将相关信息转大使馆进一步核实。

如不幸上当受骗,应及时向法国警方报案,并同时向国内公安机关报警。

受害人无法直接向国内公安机关报案的,可通过国内近亲属及时报案,并向国内报案地反电信网络诈骗中心请求帮助(拨打110即可)。

在法国还会经常有推销性质的电话,一般来电显示是08开头,讲话者或多或少有一定口音,可以直接挂断。

\subsubsection{其他诈骗}
求职、换汇、盗号、回国带货等等。

\subsubsection{个人隐私安全}
网站众多,填写时应谨防个人信息泄露。时常更换密码。区分商业广告与工作生活邮箱。

防诈骗参考:
\begin{itemize}
    \item \href{http://www.amb-chine.fr/chn/sgxw/t1823322.htm}{AMB: 中国驻法大使馆在此提醒在法中国公民谨防电信诈骗}
    \item \href{https://www.ecentime.com/article/bien-vivre-en-france}{ECENTIME: 套路太多,防不胜防!法国留学生们请收下这篇防骗指南!}
\end{itemize}

\subsection{法国安全指南}
\begin{itemize}
    \item \href{https://www.ecentime.com/article/security-guide-in-france}{ECENTIME: 法国安全参考}
    \item 
    \href{https://www.ecentime.com/article/securite-arrondissement}{ECENTIME: 巴黎地区安全参考}
    \item \href{https://mp.weixin.qq.com/s/EOGbefPAPmCXaiPJtq32Gw}{可能是全法最详细的巴黎地区治安指南}
\end{itemize}

\subsection{紧急联系方式}

欧洲急救电话:112

报警电话:17 

火警电话:18 

SAMU急救电话:15 

心理热线SOS Amitié:法语 09 72 39 40 50 / 英语 01 46 21 46 46

中毒急救电话:Paris 01 40 05 48 48 / Lille 03 20 44 44 44

驻法国使馆领事保护与协助电话:+33 1 53 75 88 40

驻马赛总领馆领事保护与协助电话:+33 4 91 32 00 19

驻里昂总领馆领事保护与协助电话:+33 7 85 62 09 31

驻斯特拉斯堡总领馆领事保护与协助电话:+33 6 09 99 44 64

驻圣但尼总领馆领事保护与协助电话:+262-693925807

驻帕皮提领事馆领事保护与协助电话:+689-87295620

外交部全球领事保护与服务应急呼叫中心电话:+86-10-12308或+86-10-59913991

驻法国使馆网址:\href{http://www.amb-chine.fr/chn/}{http://www.amb-chine.fr/chn/} 

中国领事服务网网址:\href{http://cs.mfa.gov.cn/}{http://cs.mfa.gov.cn/} 

\newpage
\section{参考网址与账号}

\subsection{综合类资讯}
新欧洲:\href{http://bbs.xineurope.com/}{http://bbs.xineurope.com/} 

一分钱: \href{https://www.ecentime.com/}{https://www.ecentime.com/} 

综合资讯公众号:巴黎疯人院、法国学生汇

新闻消息公众号:旅法华人

巴黎活动资讯网站Sortir à Paris:\href{https://www.sortiraparis.com/en/}{https://www.sortiraparis.com/en/}

\subsection{文体类资讯}

巴黎电影资讯:\href{https://paris-cine.info/}{https://paris-cine.info/}

巴黎文化艺术类资讯:公众号“八梨行动手册”

旅行推荐:公众号“法国旅游发展署”

中文书籍:\href{https://www.librairielephenix.fr/}{凤凰书店Librairie Le Phénix}、\href{https://wulolife.com/}{五楼书店Wuloulife}

\subsection{饮食类资讯}

寻味巴黎:\href{https://www.newsavour.com/restaurants/}{https://www.newsavour.com/restaurants/}


\end{document}
